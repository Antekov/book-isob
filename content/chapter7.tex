\chapter{Прибытие}
\lettrine{Т}{}ом проснулся. Он сидел в беседке посреди благоухающего сада. Казалось, что бесконечное количество разных растений окружали его, находящегося в небольшой постройке из красного дерева. Строениесочетало в себе множество стилей: от японских храмов и готических замков до чего-то еще, не вошедшего в использование человеком, но при этом оставалось близким, узнаваемым. Где-то на периферии одиноко падали лепестки цветущих деревьев. Человек, сидящий в беседке, завороженно наблюдал за всем действом. 

– Как тебе? – раздался женский голос где-то позади.

Том поднял ноги и грузно спрыгнул на пол, ушибив ступни, разведя руки в стороны. 

– Как и всегда, – усмехнувшись, ответил он, разворачиваясь к источнику голоса.  

Повернувшись, он встретился с улыбающимся лицом девушки, что оперла голову на сложенные руки, лежащие на спинке скамьи. 

– Привет, – сказал Том. – Давно не виделись, хех. 

– Привет... – глаза девушки смотрели прямо в зрачки человека, стоящего пред нею. Казалось, что она смотрит в душу, видя всё, что должно было быть скрыто. 

Неловкое молчание повисло в воздухе, создавая ощущение пальца, лежащего на курке пистолета, направленного в неизвестном для всех направлении. Но Том знал, что ему делать в данный момент. Он вышел из беседки и направился в сторону загадочно молчащей девушки. Его руки оплелись вокруг нее и так они стояли... молча обнявшись. Держась друг за друга посреди сада.

\begin{center}***\end{center}

– Кхм-кхм... а это имеет отношение ко, кхм, всему происходящему? – перебил рассказ Майкл. 

– Нуу... если быть честным... не совсем, – чеша затылок, отвечал Том. Лицо его выражало растерянность.

– Гм... может, это может подождать? Мы же еще встретимся после этого, да?

– Да... конечно... кхм... тогда я пропущу этот эпизод, перейдя к его концу, – сказал старший брат под вопросительным взглядом младшего.

\begin{center}***\end{center}

– Это твоя вина! Ты виноват во всем! – девушка кричала, она была чрезвычайно расстроена.

– Я... я, – Том мямлил, не зная, что ему сказать. Как вдруг все затянуло белой пеленой тумана, и спустя мгновение он снова был в самолете.

– Том! Кажется, вам приснился кошмар, просыпайтесь! – доктор Нуотс тряс его за плечо, обеспокоенно смотря на все еще сонный взгляд пассажира.  – Мы уже прилетели.

Все еще нервничающий после кошмара человек выглянул в окно. Заснеженный лесной пейзаж до самого горизонта предстал пред ним, и он облегченно выдохнул. 

– Возьмите пальто, на улице минус, – Тому протянули элемент теплой верхней одежды на вид чуть больше необходимого. 

Ледяной ветер атаковал своими порывами сразу после выхода из самолета на трап. Что-то неуловимое ощущалось в этом ветре и окружающем воздухе. Казалось, что он проникал прямиком в разум сквозь ноздри, щекоча давно забытые вещи, идеи, страхи. Словно ураган проносился в голове, сметая абсолютно все без исключений в один котел мыслительного процесса. Что порождало самые различные идеи, которые могли довести до безумия, если бы были постоянными. Но здесь они были лишь за секунду убегающей строкой в огромной неоновой вывеске сознания. 

– Прошу за мной, – сказал впереди идущий Нуотс в своей, видимо, привычной манере: немного приторной и фальшиво дружелюбной. Он продвигался вперед, забавно сопротивляясь встречному ветру. Казалось, что он вот-вот подлетит и очутится на месте воздушного шарика, что безрассудно запускают в небо дети на различных празднествах. 

Том молча следовал за доктором, оглядываясь, пытаясь увидеть что-то чрез почти закрытые веки. Они шли к монструозному и угрюмому серому зданию. Оно практически сливалось со всем грязно-белым окружением, лишь немного выбиваясь среди массива леса. Уже подойдя вплотную, Том заметил небольшой вход. «Тут были бы так кстати огромные деревянные врата», – с усмешкой подумал он. Нуотс подошел к небольшой коробочке слева от металлических дверей со стеклянными вставками на уровне глаз и поднес карточку, извлеченную из его кармана. Прозвучал приветственный писк и над входом загорелась зеленая полоса. Они зашли внутрь. 

Мерные шаги звучали в пустынных коридорах. Металлические звуконепроницаемые двери с номерами разбавляли пустые стены вместе с редкими растениями в горшках. 

– Куда мы идем? – Том разрушил ритмичные гулкие звуки, задав вопрос.

– Ах, точно. Я совсем забыл вам сказать, – сказал Нуотс с небольшим смешком. – Я веду вас в ваш кабинет. Как только вы обустроитесь, придет мой ассистент. Думаю, он вам понравится, – сказал доктор, подмигнув Тому. – Впрочем, мы уже пришли. Вот ваш кабинет.

– Спасибо, – немногословно ответил собеседник. – Дальше я сам, так?

– Да-да, мне, к сожалению, нужно идти. Мой помощник придет к вам чуть позже, – уже уходя, сказал толстяк. 

Том проводил его взглядом, пока тот не скрылся за поворотом. Посмотрев на дверь, он увидел на ней номер - «42». «Хах, что же, номер мне нравится», – подумал он, собираясь открыть дверь. Но она не поддалась его попыткам. Осмотрев ее еще раз, он увидел весьма знакомую коробочку слева от ручки. 

– Оу... кажется... мне не дали ключ. Оу, – Том слух озвучил свои мысли, будто бы вел какое-нибудь телешоу. Это было своеобразной привычкой для него, что проявлялась в моменты его одиночного времяпрепровождения. Озадаченно оглядываясь, Том решил пойти за Нуотсом. Теперь уже только его шаги мерно звучали в пустых коридорах. «Черт. Как так вышло? А может стоило остаться на месте и дождаться ассистента?» – думал он, поворачивая уже на третьей или даже четвертой развилке. 

– Кхм. Кажется, я очень сглупил. Надо было хотя бы запоминать куда я поворачивал. Ох. Да что за день такой? – Том стоял уже на неизвестно какой развилке. – Не здание, а лабиринт. Где планы эвакуации хотя бы?

Внезапно из-за двери справа от него раздался приглушенный крик. Он звучал лишь мгновение, но животный инстинкт так и умолял, чтобы Том бежал отсюда. Но он не слушал его. Он положил руку на дверь и нажал на нее. Она открылась.

\begin{center}***\end{center}

«Что я, черт побери, творю?!» – думал Том, открывая дверь все шире. Слыша вопль все лучше. Но он не остановился, и дверь открылась. Это было ярко освещенное белым светом помещение. По стенам стояли разные устройства, некоторые были знакомы Тому, другие были известны ему лишь по книгам.  В конце комнаты виднелась ширма, разделяющая помещение на две части. Прямо за ней находился источник шума. Внезапно все стихло, комнату окутала тишина. Но это состояние сохранялось лишь мгновение: из-за ширмы послышался глухой хлопок и на разграничительную ткань брызнула алая кровь. Кардиомонитор пропищал свою провожающую песню. Из-за ширмы послышался вздох и звук возни с какой-то одеждой.

– Записывай, Карен: Исследуемый объект №41. Результат отрицательный. Проблемы с кровяным давлением. Произошла аневризма сонной артерии. Записала? – низкий мужской голос раз-несся по комнате.

– Да, Боб, все. Фуф. Почему это происходит? Мы, наверное, чего-то не видим. Да и Нуотс тоже, – отвечало лирически легкое сопрано. 

Ширма начала подниматься, кто-то выходил из той части комнаты прямо к Тому. А он ошеломленно стоял, не зная, что делать. Поэтому он был недвижим и молча корил себя за свою глупость. Но тут его посетила идея, что лучше как-то заранее уведомить о своем присутствии, чтобы не напугать теоретических коллег. 

– Экхм, – начал он с легкого кашля. – Приветствую! Я тут....

– ТВОЮ ЖЕ... – мягко говоря, мужчина был очень удивлен нежданному гостю. Он отпрянул в сторону и споткнулся о стоящий рядом стул. Надетый на нем костюм химической защиты и шлем в руках явно не давал ему особой маневренности.

– Боб, что там у тебя?! – взволнованно спросила женщина, преодолевая ширму. – Вы еще кто такой? – уставилась она на Тома.

– Я немного... – его опять прервали.

– Вот и я тем же вопросом задаюсь, Карен. ЧТО ТЫ ТУТ ЗАБЫЛ И КАК СЮДА ПОПАЛ? – вставая и попутно кряхтя, спросил его встающий. А, по всей видимости, Карен двигалась куда-то вправо с не совсем понятыми для Тома целями. 

– Я новый сотрудник. Я от доктора Нуотса. Он провел меня к кабинету, но забыл дать мне ключ-карту, – нежданный гость немного поднял руки вверх, чтобы как-то показать свою безобидность. 

– Вы от Нуотса? Ах. Черт. Я же должна была прийти к вам. Совсем вылетело из головы. Вы же Том Браун, так? Прошу прощения, тут, кхм, – она посмотрела на ширму. – Возникли некоторые трудности. Но почему вы не остались ждать у кабинета? – она вопросительно посмотрела на Тома своими голубыми глазами, в которых можно было утонуть даже через еще не снятый шлем. Впрочем, они не помешали ему ответить на ее вопрос.

– Пожалуй, любопытство сыграло со мной злую шутку, – сказал нежданный гость. – Так вы меня проведете?

– Да, не волнуйтесь. Дайте только переодеть-ся, ждите меня в коридоре, – она пошла в сторону шкафа, стоящего у самого входа. 

Том же вышел из комнаты, услышав, как Карен что-то сказала Бобу. Но что она произнесла, он уже не разобрал. В коридоре он почувствовал до боли знакомый запах, который он игнорировал раньше. Это была смесь больницы, стоматологической клиники с толикой идущей стройки. «Странное сочетание», – думал он, облокотившись на стену. «Исследуемый объект №41... хах... я мог бы увидеть здесь некую символичность, что мне выделили кабинет сорок два и сразу же после моего приезда умер объект сорок один. Но нет. Это лишь совпадение, а совпадения происходят постоянно, – мысли Тома были хаотичны и будто бы с нарушением логики. Он и сам это чувствовал, но оправдывал легким шоком после увиденного. – Надеюсь, что в моем кабинете есть кулер, а то так хочется пить...» – так он и размышлял, отстраняясь от мыслей о своей будущей работе, но тут дверь открылась и оторвала его непринуждённых раздумий.

– Так, ну вот. Гм. Я так и не представилась. Я Карен Морган – ассистент уже знакомого вам доктора Нуотса. О, а вот и ваш ключ, – она по-смотрела Тому за спину и приветливо улыбнулась. – Спасибо, Барри! – Том развернулся и увидел статного мужчину в серой форме: видимо, охранника.

– Да не за что, – грубым басом, содержащим некий особый теплый оттенок, ответил он и отдал ей карту. – Все? 
– Да-да, спасибо еще раз, –  не снимая улыбку с лица, сказала Карен, и он пошел в обратную сторону по этим коридорным венам здания. 

– Ну что, – она посмотрела на Тома. – Пойдемте, – сказала женщина, уже шагая вперед. 

Она шла весьма торопливо. У некоторых людей это было своеобразной привычкой после житья в больших мегаполисах, где люди постоянно торопятся и пребывают в движении. Другие же приобрели ее за счет работы на больших производствах в качестве руководящей должности. «Интересно, к какому из этих типов относится Карен?» – думал Том, рассматривая ее спину, за-крытую белым халатом. На этой белейшей ткани не было ничего, кроме одного бейджа на веревочке с фотографией, именем и каким-то номе-ром. Под застегнутым халатом же, можно было увидеть только обычную одежду. Единственное, что можно выделить, так это то, что вся она была облегающей, дабы не мешать при работе, как ду-мал Том. Ведь он сам придерживался такой одежды.   

– Все. Вот и ваш кабинет, – они остановились под уже знакомым Тому номером. – Вот ваша ключ-карта, вы прикладываете ее вот сюда, – она отдала ему карту и показала тонкой ладонью на ту самую коробочку. – И дверь открывается. Также на обратной стороне ключа находится информация о вас. Поэтому могу порекомендовать повесить ее на шею, как бейдж, – она перевела руку на свою грудную клетку, на уровне которой находилась ее карта. – Довольно удобно, как по мне. 

– Понял. Хорошо, – Том приложил свой ключ, и зеленая лампочка в правом верхнем углу косяка просигнализировала, что дверь открылась. 

– Ой, подождите. Смотрите, по коридорам есть небольшие полусферы на стенах. Если приложить карту к ним, вам откроется карта здания с отмеченными местами, в которые вам, может быть, надо идти. На случай, если вы заплутаете, в общем. Теперь точно все. Заходите, – она протянула руки к двери, как бы приглашая Тома зайти внутрь. 
\clearpage
{\begingroup
\ThisCenterWallPaper{1.0}{после 7 чаcти.jpg}
\noindent
\endgroup}
\cleardoublepage