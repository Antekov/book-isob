\chapter{Спутник}
\lettrine{О}{}н пришел на кухню, чтобы приготовить что-то поесть. В хижине не было часов, доставать телефон у него не было желания, но внутренний голос уже требовал какой-нибудь еды. Нельзя сказать, что Майкл очень хорошо готовил, но разные обстоятельства требовали широкого спектра умений, так что прокормить себя сносной едой он всегда мог, а тут даже условия были довольно неплохими. Он не долго думал о том, что ему приготовить. В шкафах он видел макароны и тушенку, где-то в глубине полки были различимы грибные консервы. «Самое время поэкспериментировать» — думал он, доставая все продукты. 

Расположив все на внушительной столешнице гарнитура, он отправился на поиски посуды. «Черт, с пистолетом как-то не очень удобно, надо выложить его. Да и вряд ли что-то случится», — думал он, уже кладя оружие на кухонный подоконник. Спустя несколько секунд первые пузырьки воздуха уже всплывали с дна кастрюли, а в сковороде шипела тушенка. «Неужто тут не будет никаких овощей?» — думал Майкл, осматривая будто бы бесконечные шкафы, изредка отрываясь на помешивание содержимого посуды. Уже отчаявшись найти хоть что-то, он обернулся и рядом с дверью увидел три потрепанных белых мешка, которые немного расползлись в нескольких местах. В них он наконец-то нашел то, что искал: картошка, морковка, лук. «Странно. Я всегда сразу анализировал любую комнату, в которую входил, а сейчас я словно в тумане. Сейчас еще рано расслабляться. Я все еще Майкл, а не простой городской парень со спокойной жизнью» — размышлял он, добавляя лук и морковку в сковороду. «Надо было сначала найти овощи и обжарить их, а то сейчас у меня весь порядок приготовления сбит, — негодовал он. – Да и в камин стоит еще дров добавить, а то те уже почти полностью стали пеплом, когда я уходил. Но тогда нужно выходить на улицу, хотя, возможно, что с таким уровнем заботы о клиенте, где-то рядом с домом будут уже наколотые дрова... хм, чем это пахнет... ТУШЕНКА!!!» — Майкл экстренно снял с конфорки электроплитки сковороду с чуть подгоревшим содержимым. Спустя некоторое время на столе уже стояла кастрюля с готовым блюдом, притягивающим к себе паром, растворяющимся в воздухе. Сам кулинар же стоял перед своим творением, скрестив руки на груди как минимум удовлетворенный результатом. «Что там с этим «пугалом», интересно» — думал он, выходя из кухни.

В зале его встретил Исоб все в той же позе и яркий белый свет из окна – метель наконец-то улеглась и свет, многократно отражаясь от снега, будто бы хотел ослепить все, что его окружает. Немного пройдя в глубь комнаты, в поле зрения Майкла попал браслет. Он уже не горел красным. Зеленый ярко светил, будто бы внушая спокойствие. Но Майклу от этого стало не по себе, предчувствие кричало что-то, но он не мог разобрать. Он подошел к Исобу и вновь попробовал его разбудить. Сильно сжав его плечо, он начал раскачивать недвижимое тело.

Только он собирался оставить попытки, как за его спиной раздался столь знакомый ему звук разбивающегося стекла. Этот звон он не мог ни с чем спутать. Прямо сейчас, на потертый деревянный пол, что давно следовало бы отшлифовать, падал небольшой объект, похожий на цилиндр. Время словно прекратило свой ход. Мысли же в голове Майкла, напротив, неслись с огромной скоростью: «Так. Это. Граната. Какая? ЧЕРТ! Убивать меня бы никто не стал, т.к. я нужен всем живым, — цилиндр уже прошел половину пути до поверхности. – Следовательно, это либо газ, либо светошумовая, — до пола осталось совсем немного. – Пока не произошел взрыв, я должен лечь и закрыть уши с глазами» — Майкл бросился на пол, и только в падении вспомнил про Исоба: «Извиняй, дружище... Тут я тебе не помощник». Комнату озарила вспышка, звук ударил по барабанным перепонкам. «Раз... два...три... четыре... пора!» — Майкл рванулся к спальне, перепрыгнув чрез место, где предположительно лежала взорвавшаяся граната. Ворвавшись в комнату, болезненно врезавшись в дверь, Майкл открыл глаза. Он видел. Но голова гудела, в ушах стоял дикий звон, из-за чего ориентироваться было довольно затруднительно. Но он продолжал движение. Бросив взгляд на окно и убедившись, что оттуда не начнут огонь, он попытался перевернуть кровать. Из зала донеся звук бьющегося стекла и очередного падения предмета чуть тяжелее небольшого камня. Продолжая поднимать кровать, Майкл захлопнул дверь ногой. Прозвучал взрыв. Вспышка не затронула его, но звуковая волна вновь болезненно прошла по его ушам: «О боже! Черт! Проклятье!» — крутилось в его голове. Наконец поставив кровать на бок, он увидел небольшой ключ, приклеенный скотчем к бортику постели. «Какая халтура!» — успел подумать он прежде, чем вставить спасительный предмет в замок и провернуть его. «Боже, кто придумал такой метод защиты?! А если бы по мне уже стреляли? Как бы я открыл этот чертов шкаф?!» — негодовал он, стаскивая цепь с шкафа. Покончив с ней, он открыл шкаф. В нем стоял сейф с подписью «Бок гарнитура, кухня». 

— ЧТО ЭТО ЗА ******* КВЕСТЫ?! – выкрикнул он, в порыве ярости. Слева от сейфа он увидел револьвер с патронами, что хоть как-то его успокоило. Но ненадолго. Из зала вновь донесся звук удара об пол. Заряжая барабан револьвера, Майкл уже приготовился к третьему звуковому удару, но его не последовало. Лишь тихое шипение доноси-лось из комнаты, что он покинул. «Проклятье, они закинули газ. Не смертельно, но в зал мне уже не стоит соваться. Странно, почему они не атакуют дом со всех сторон, ведь... — его размышления были прерваны звоном стекла и упавшей перед ним светошумовой гранатой – ДА ЧТОБ ТЕБЯ!». У него не было времени на раздумья. Остаться здесь – возможно лишиться зрения и слуха или даже жизни. Выход был, но не факт, что он был спасением. Из зала донеслись тяжелые шаги. «НЕТ ВРЕМЕНИ РАЗМЫШЛЯТЬ!» — внутренне прокричал Майкл и прыгнул в разбитое окно. Вслед за его прыжком последовал взрыв. Осколки стекла вонзились в него по всему телу. Барабанные перепонки, казалось, лопнули и все, что заполняло его голову – сплошной звон. Рубашка постепенно пропитывалась кровью, сменяя свой белоснежно-белый цвет на бордовый. Майкл рухнул в снег, стекло еще глубже пронзило его плоть, а боль окатила его новой волной. «Нельзя лежать. Я должен встать. Я НЕ МОГУ ТАК ЗАКОНЧИТЬ. Я СЛИШКОМ БЛИЗОК» — еле-еле оставаясь в сознании мыслил он. Он поднял голову и увидел троих людей. Двое в форме, напоминающее специальные войска. Их одежда была черной и очень сильно выделялась на удивительно белом снегу. Между ними стоял человек в расстёгнутом пальто, из-под которого виднелся классический костюм. На нем не было маски, но Майкл не мог различить его лица.

— Его брать? – спросил один из людей в форме.

— Да, – сухо ответил ему человек в пальто, разворачиваясь и уходя в сторону дороги. 

Майкл хотел поднять револьвер, но рука не слушалась его. Мир начал расплываться в его глазах, с краев наступала тьма. Он теряет сознание. Он знал это. И это страшило его, он боялся, что уже не откроет глаза, а последнее, что он увидит, будет то, как кто-то уходит. Он близок к смерти. И это было уже не в первый раз, но сейчас он уже не верит, что сможет выкарабкаться...

\begin{center}***\end{center}

Майкл плыл в этой тьме. Все его раны ушли, а рваная одежда стала вновь словно с прилавка. Ничто не тревожило его. Ничто уже не волновало. Это будет последнее плавание, он был уверен в этом. Он поднял руку и указал вверх. В это же мгновение будто бы со всех сторон послышалось падение капли в воду. Четкий звук пронесся по бесконечному пространству.

— Ну что? Это уже все? – вопрошал Майкл в пустоту.

— А ты как думаешь? – спросил его неожиданно появившийся откуда-то слева темный силуэт в балахоне. Он словно плыл по воздуху. Все, что было под тканью, скрывалось густой и тягучей смолью мрака. Лишь костяная рука без плоти была открыта взору Майкла. Она сжимала такую знакомую ему пачку.

— Хочешь? – протянулась за секунду материализовавшаяся сигарета в сторону все также безмятежно лежащего ни на чем Майкла.

— Я бросил, – прозвучал ответ.

— Ты бросил? Какие глобальные перемены! Вот уж не думал, — восклицал силуэт, поднося уже зажженную сигарету к области, в которой, предположительно, находилось лицо. – Хах! Бросил! С чего бы? Как давно?

— Да знаешь, решил, что хватит травить себя. Решил и сразу бросил, – невозмутимо говорил Майкл.

— Так как давно?

— Часов 5 назад, — с усмешкой ответил он.

— Ха! Я в тебе не сомневался, — смеясь, произнес силуэт уже без сигареты меж пальцев. Казалось, что дружелюбная атмосфера полностью заполнит собой пространство, но странное зудящее ощущение никак не покидало Майкла.

— Слушай, так я умру сейчас? – спросил он, повер-нув голову в сторону своего собеседника.

— Боже мой, ты снова об этом? Каждый раз, когда мы видимся, ты спрашиваешь одно и то же, а ведь это и так происходит довольно редко. Никакого разнообразия! Нельзя было как-то понять, что мне это неизвестно. А если известно, то я бы тебе не сказал, — раздраженно звучал голос из-под балахона.

— Не сказал бы? Но почему? – сложив руки на груди, сказал Майкл с едва уловимой обидой в голосе.

— А зачем тебе это знать?  Если я скажу тебе, что ты сейчас умрешь, ты будешь рад или спокоен достаточно, чтобы поддержать разговор?

— Ну как зачем. Это же естественно. После смерти меня же будет ждать что-то, а я бы не хотел пропустить момент, когда это что-то уже будет здесь! 

— Сомневаюсь, что можно пропустить момент собственной смерти. Даже при твоей рассеянности в бытовых моментах, — голова, скрытая под капюшоном, повернулась в сторону своего собеседника. Майкл не видел лица, лишь тьму, но ему казалось, что лик сущности озарился странной улыбкой.

— Возможно, и так. Но что я тут тогда делаю? — спросил он, внимательно всматриваясь в пространство под тканью.

— Охх... Ты же знаешь ответ, только не надо его искать сейчас. Расслабься в кои то веки и просто насладись тишиной. Лучше вот, смотри, что я теперь могу, — с легкой, чуть ли не учительской, надменностью произнес собеседник Майкла, занося костяную руку и складывая пальцы в положение для щелчка. И уже секунду спустя прозвучал громкий, казавшийся невозможным для выполнения лишь на костях, щелчок. Оглушительной и одновременно еле уловимой волной пронесся его отзвук. Майкл закрыл глаза. Он не знал, что сейчас произойдет, но решил внутренне подготовится ко всему. «Глубокий вдох... Еще один... Боже, какой чудесный запах!» — Майкл открыл глаза. Вся тьма ушла и на ее место пожаловала тихая лесная поляна. Он лежал в траве. Она была мокрой, но это его ничуть не волновало. С неба падали небольшие капли теплой воды — начинался дождь. Майкл втянул в себя свежий и бодрящий воздух, набрав полную грудь. Майкл ощущал столь знакомое и так давно забытое чувство. Он будто бы вновь оказался в своем далеком-далеком прошлом. В детстве. Повернув голову, он вновь увидел своего спутника. Тот стоял в позе, явно говорящей о том, как он горд своей работой: руки сложены на груди, а голова, все также скрытая капюшоном, была приподнята.

— Ну как тебе? — с явным удовлетворением спросил он.

— Неплохо, весьма неплохо, — вновь набрав воздуха, отвечал Майкл. Внезапно где-то позади Майкл услышал смех. Он был громким и беззаботным, так и говорящий о том, как же весело его источникам. Вопросительно взглянув на своего спутника, Майкл начал вставать, чтобы узнать, кто же врывается в его самое безмятежное время.

— Майкл! Стой! Ты же знаешь, что я тебя догоню! – выкрикивал один из бегущих мальчиков лет девяти, пыта-ясь догнать другого, немного помладше на вид.

— Так ты поймай сначала, Том! – весело отвечал ему второй, продолжая свой бег. Оба они были одеты в обычную одежду: шорты, черные футболки с логотипом какой-то группы.

— Пфф... Ты поместил меня в мои воспоминания? Как оригинально! Аплодирую стоя! — с упреком сказал Майкл, глядя на темный балахон, саркастически хлопая. Спутник его потерял былую уверенную позу. Майкл вознес руки вверх и пнул траву.

— Тебе не нравится? Я, вообще-то, старался! Знаешь, это не так просто, как может показаться! — отвечал капюшон. Голос его смешивал в себе целый ряд эмоций: от грусти с обидой до раздражения и злобы.

— Ну как тебе сказать... — громкий крик прервал Майкла: маленький Том все-таки догнал убегающего мальчика и теперь они боролись в мокрой траве. 

— Если я сейчас умираю, то это как-то сомнительно. Я обещал Тому... старшему брату... что стану на путь исправления и завяжу со всеми нелегальными подработками... нелегальной работой, точнее...  И я же почти это сделал. Почти, – глаза его стали влажными, а руки непроизвольно сжались в кулаки. Майкл терпеть не мог подводить брата. Где-то на фоне продолжали, смеясь, возиться его воспоминания.

— Я был так близок к тому, что хотел для меня Том. У меня были деньги, чтобы полностью устроиться в новом городе. Я бы приехал в новый город и купил бы там квартиру, устроился в адвокатскую контору, не зря же получал образование. Купил бы велосипед и на нем доезжал до работы. Познакомился бы с кем-нибудь. Завел бы друзей. Нашел бы девушку, ту самую. Сделал бы ей предложение испытав все эти стандартные трудности. Я бы жил так, как хотел Том, как хотел бы \textit{Я}! – Майкл, прервав свой монолог, опустился на колени и подставил дождю лицо. Теплые капли потекли по его лицу. Он закрыл глаза.

— ... 

— Легкие деньги вскружили мне голову, и я вошел в игру, из которой уже нельзя выйти. Работа на картель, наркоторговля, даже убийства... ничто не было мне чуждо. Но все время, абсолютно все, что-то свербело внутри меня, не давая покоя, — Майкл опустил голову и открыл глаза. Его колени были немного погружены в обжигающий песок, позади слышался звук ссыпающейся почвы и копающей лопаты. Укоризненно посмотрев на все еще висящий в воздухе балахон, Майкл повернул голову налево — там он увидел черный мешок. Повернувшись, он встретил себя, в поте лица копающего могилу или же, скорее, просто яму. 

— Спасибо, что хоть без запаха, он был весьма сомнительным, — Майкл перевел взгляд на своего спутника и увидел, как тот держит пальцы, готовые к щелчку. — Нет-нет, даже не думай! — костяная рука опустилась.

— А нельзя вернуться куда-то в более приятное место? Я тут веду свой монолог, рассказываю тебе все, а ты взял и перенес в это ужасное место. Несправедливо, не находишь?

— Эээ... да, конечно. Закрой глаза, — проговорил голос из-под капюшона. Майкл повиновался и уже в следующее мгновение услышал, как волны мерно накатывают на песчаный брег.

— Благодарю, — он открыл глаза и увидел бесконечный океан, что распростирался до самого горизонта. Солнце приятно ласкало кожу, а шум воды убаюкивал и успокаивал. «Здесь я и хотел умереть, — подумал Майкл. — И вот сейчас, я уже почти покончил со всем. Смог убежать, купить себе «новую личность», «новую жизнь». Но что-то пошло не так. Я даже не знаю, откуда были те парни и как я им насолил, понимаешь?»

— Да, конечно, понимаю, — дружески произнес парящий балахон, повернувшись в сторону океана. Майклу казалось, что он будто бы погрустнел и стал еще темнее, чем раньше.

— Вот такая история. А еще этот странный парень... Исоб. Интересно, он еще жив? Очень сомневаюсь, конечно, но вдруг. Надеюсь, что хоть у него все сложится лучше, чем у меня. — с грустной усмешкой произнес Майкл, всматри-ваясь в даль. На песчаном пляже сидел человек, а рядом с ним парил балахон с костяной рукой, выглядывающей из-под ткани. Каждый из них выглядел обеспокоенным, но при этом возникало ободряюще согревающее ощущение, что все будет хорошо. Дружный вздох окатил пляж...

— Ты можешь сказать мне, что находится под тканью? Я как-то не успевал просить тебя об этом ранее, — вопрошал своего спутника Майкл 

— Нет. Но я могу тебе показать, хочешь? Только сразу предупрежу, тебе это может не понравиться, — сказал голос, повернувшись темным пространством капюшона к сидящему.

— Хорошо. Давай, — подумав, ответил Майкл.

Костяная рука взялась за верх капюшона и начала медленно отводить ткань. Тьма, скрывающая лицо, начинала медленно рассеиваться. Спустя мгновения Майкл увидел... себя. Искалеченного, поврежденного. Плоть словно отслаивалась с его искаженного лица, а один глаз был закрыт повязкой. С правой стороны лица была дыра, сквозь которую просматривались зубы. Глаз же, что оставался виден, смотрел на Майкла с бесконечной печалью и болью. 

— Ты... ты, — Майкл не мог подобрать слова. Он был ошарашен и напуган. 

— Да. Ты все правильно понял, — лицо, скрывавшееся под тканью, исказилось в грустной ухмылке.
Майкл не мог произнести и слова, тишина охватила пляж.

— Ты все правильно понял, — вновь повторила его копия. – Но сейчас пора просыпаться и делать то, что ты так хотел! – единственный глаз уставился на Майкла, а в костяной руке появился шестизарядный револьвер. Не успев что бы то ни было сделать, пистолет оказался прямо перед лицом все еще сидящего и обескураженного Майкла. 

— \textit{Вставай!}\\[0.5\baselineskip]

БАХ

\clearpage
{\begingroup
\ThisCenterWallPaper{1.0}{после 3 части.jpg}
\noindent
\endgroup}
\cleardoublepage
