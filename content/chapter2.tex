\chapter{Кофе}
\lettrine{К}{}омнату вновь наполнило теплой, как будто поглощающей, атмосферой. Вопросы все также крутились в голове у Майкла, но задать он их будто бы не мог. Словно что-то останавливало его от этого. Так они и сидели – молча. Никто из них не проронил ни слова, каждый был погружен в свои мысли. В глазах Исоба плясало отражение пламени, а в глубине зрачка словно читалась тихая грусть. Постепенно он засыпал, погружаясь все глубже в себя. Майкл думал, что не сможет заснуть в эту ночь уж точно, но уже спустя несколько минут он отправился в царство Морфея вслед за Исобом. На удивление, Майкл спал очень крепко, как не спал уже много лет.

В комнате, которая выглядела одновременно просторной и тесной, спало два человека. Один лежал на диване, укрываясь курткой, а второй кое-как приютился на стуле. Угли в камине приветливо светили своим ярко-красным светом, а за окном все мело и мело. Лампочка продолжала висеть, уходя проводом в мрак потолка, а кабан с камина величественно возвышался над всей комнатой, будто она принадлежала только ему, и он милостиво приютил двух странников на время бури. Человек на стуле начал просыпаться, открывать глаза. Он окинул комнату сонным взглядом, будто бы вспоминая, что же он тут делает. Он посмотрел на все еще спящего и тихо усмехнулся себе под нос. Вопросы так и не покинули его, ведь еще с детства любопытство было верным спутником Майкла.

Он тихо поднялся, словно боясь разбудить своего гостя. Хотя, подумав, он понял, что действительно не хочет его разбудить. Он развернулся и вновь увидел так и не открытую им дверь. Еле слышно он продвигался к ней и, открыв, он не увидел ничего сверхъестественного. Это была довольно пустая комната с двуспальной кроватью. В противоположной стене виднелась открытая дверь, за которой, по всей видимости, была ванная комната. Майкл поспешил в нее и через пару минут вышел весьма довольным на вид. Выходя, он, наконец, обратил внимание на большой шкаф прямо напротив кровати. Он был обмотан цепью, сцепленной замком, а на дверце висел небольшой бумажный обрывок. Подойдя ближе, он прочитал: «На крайний случай! Кровать». Майкл догадывался, что лежит в шкафу, и он очень надеялся, что этот случай не наступит. «Однако, если он произойдет, товар этого «гиганта» должен очень помочь мне», — думал он, осторожно проходя мимо. Майкл вышел из комнаты. На диване все еще спал Исоб, а от углей осталась лишь пара красных точек. Так как другого источника тепла в хижине не было, воздух начинал постепенно остывать, но атмосфера в зале все оставалась душевно мягкой. Он взял оставшиеся три полена в руки, еле-еле удерживая их. Уже у самого камина он чуть не уронил одно из них себе на ногу, но в последний момент смог прижать его к себе правой рукой. Майкл оглянулся на диван, но Исоб не подавал каких-либо признаков того, что его это хоть как-то потревожило. Успокоенный, водитель начал укладывать чурки в камин. Пока он занимался этим, его голову посетили мысли о том, почему же он вдруг так заботится о Исобе. Будто он приходится ему не должником, а, напротив, будто сам Майкл обязан ему. Впрочем, эти мысли быстро покинули его, сменившись на приятное удовлетворение. Майкл очень заботливый человек, хоть сам он и не хочет, а возможно, просто не может, этого признать. Даже обозначить это для самого себя.

Закончив с огнем, Майкл отправился на кухню с надеждой, что хотя бы в одном из шкафов будет кофе. К его счастью, в гарнитуре было несколько пачек отличного молотого кофе, турка и записка с текстом: «По особой просьбе». Майкл был просто счастлив: «Вот это сервис!» — весело прошептал он. «Клиент» давно не ощущал подобного спокойствия. Казалось, что все его тело и мысли вот-вот расплывутся в нирване. Его движения приобрели элегантную легкость и плавность. Кофе медленно ссыпался с чайной ложки прямиком в турку. Майкл смотрел, как вода из бутылки льется, заливая молотый черный порошок. Поставил кофе на небольшую электрическую плитку. «Да как тобой пользоваться?» — думал Майкл, задумчиво смотря на ручки и кнопки. После некоторой возни с плитой, напиток начал нагреваться. А зрелый бариста, следуя своей привычке, полез в карман за сигаретами. Достав мятую упаковку с огромным красным яблоком, Майкл просто уставился на нее. Руки хотели открыть ее и достать сигарету, но разум его, словно бил тревогу, отвергая всякую мысль о том, что-бы затянуться и выпустить в комнату несколько клубов дыма. «Ну и черт с ними. Всегда хотел бросить. Если уж меняться, то кардинально», — думал он. Но пачка все же отправилась на свое привычное место – прямиком в карман синих, немного свободных джинсов.

Пока кофе плавно подходил к температуре кипения, Майкл начал рыскать по шкафам в поисках чашки. В поисках чашек. Он не знал, будет ли Исоб кофе, но, если нет, кому-то в этой хижине достанется две чашки отличного напитка, а против этого Майкл ничего не имел. Аккуратно взяв чашки, он отправился в общий зал. Открывая дверь боком, он уже видел, что Исоб сидел на диване, все так же удивленно наблюдая за огнем.

— Любишь огонь? – спросил Майкл.

— А? – нервно обернувшись, сказал Исоб – Огонь? Ты про этот теплый свет?

— Эм... да. Ты же знаешь, что такое огонь? Возьми, – протянул кружку с напитком.

— Я... тут... столько всего... странного. Я не понимаю, – тихо проговорил Исоб.

— Что именно ты не понимаешь? – неожиданно так же тихо спросил Майкл.

— Где я? Где друг? – Исоб растерянно посмотрел на собеседника. – Где другой друг?

— Я не знаю. Кто ты? Откуда ты взялся на дороге в лесу посереди ночи? – отделяя каждое слово, будто бы ста-нок, печатающий детали, произнес «не тот друг». Но ответа не последовало. Глаза Исоба застыли и, развернувшись, он словно впал в транс. Майкл никак не ожидал такой реакции. Ни его слова, ни легкая тряска за плечо не смогли разбудить Исоба. Правильный профиль его бледного лица предстал пред Майклом и казалось, что ничто не сможет разбудить его, будто он внезапно перевоплотился в древнегреческую скульптуру Каламида. Бросив свои попытки разбудить собеседника, он попытался вытащить кружку из крепко сжатой руки. Аккуратно, разгибая палец за пальцем, он наконец забрал приготовленный им напиток, что уже успел немного остыть. Но тем не менее, Майкл с удовольствием выпил и вторую кружку. «Что же ты такое?» — задавался он вопросом, глядя на застывшую «статую». Внезапно его взгляд приковало слабое мерцание чуть повыше лодыжки. Словно за джинсами находилась небольшая лампочка или светодиод, горящий красным цветом, что кое-как проходил сквозь одежду. 

— Извиняй, дружище, — сказал Майкл, поднимая штанину. Убрав мешающую ткань, он в конце концов увидел черный браслет с двумя светодиодами и белой гравировкой, что обворачивала его: «\textbf{10-24-4-10-25}».  Кроваво-красный свет падал на озадаченное лицо Майкла. Мысли окутывали его разум, он буквально тонул в них, но чем дальше погружался в раздумья, тем больше он терялся в себе: «Он... заключенный? Что он мог сделать? Да и как смог убежать, и откуда, если тут нет ничего на километры и километры? Это его номер? Как-то не похоже на все предыдущие, что я видел. Слишком много вопросов и никаких ответов!»

Дрова в камине уже почти прогорели, а буря, казалось, начинала утихать. Майкл сидел на стуле и молча смотрел то на огонь, то на Исоба. Он уже жалел, что связал-ся с ним. Неприятности вряд ли обойдут его стороной, как бы он не хотел обратного. Исоб все еще оставался в образе и будто бы не моргал. 

— Что же с тобой делать? – со странной усмешкой произнес Майкл, вставая со стула. – Не хотелось бы пятнать начало новой жизни, – добавил он, вставая. 
\clearpage
{\begingroup
\ThisCenterWallPaper{0.95}{после 2 части.jpg}
\noindent
\endgroup}
\cleardoublepage
