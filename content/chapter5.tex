\chapter{Ресторан}
\lettrine{З}{}айдя в ресторан, он посмотрел на часы – ровно час дня. Окинув помещение взглядом, он не увидел кого-либо, кто выглядел бы ждущим его. «Видимо, они менее пунктуальны», – с усмешкой подумал он. Заприметив столик вдали от всех посетителей и динамиков, он сел туда.

– Вы будете одни? – немного напугала Тома, внезапно появившаяся официантка. 

– Нет, не один. Ко мне вот-вот присоединятся, – отвечал он

– Хорошо, – дружелюбно улыбаясь, сказала официантка. – Хотите чего-нибудь?

– Эээ... да, можно, пожалуйста, минеральной воды и ристретто? – странная головная боль вернулась к нему, и он воспользовался давно открытым рецептом.  

– Конечно, – продолжая мило улыбаться, она ушла куда-то вглубь зала. Том следил за ней взглядом, пока она не исчезла из виду. 

– Том Браун? – раздался голос за его спиной.

– Да. Здравствуйте, – Том попытался встать и протянуть руку для рукопожатия, но задел стул ногой и чуть было не упал. Поднявшись, он попытался сгладить неловкое начало. – Какой я неловкий, хехе. 

– Здравствуйте, я доктор Нуотс, – сказал немного толстый, но выглядящий довольно харизматично, мужчина лет 30. Он протянул Тому правую руку, словно, не заметив только что произошедшего. В его левой руке была зеленая папка, в которую были вложены и другие. Том пожал его руку, и они сели за стол. 

– Вы говорили о каком-то предложении? – спросил он, стараясь сделать свой голос как можно более непринужденным. 

– Да. И я думаю оно вас заинтересует. Я хочу предложить вам одну работу на довольно неплохой должности, – он открыл папку и протянул Тому. Неуверенно посмотрев на протянутый объект, он все-таки взял его. Бросив взгляд на доктора, и увидев одобрение в его глазах, он открыл папку. В ней лежал один-единственный лист. Судя по всему, в самом верху была написана предлагаемая должность: «Биоинжинер». Том учился на эту должность и даже работал по профессии, но недавний конфликт с начальством лишил его работы. Опустив взгляд, он увидел условия, написанные небольшим текстом на четверть страницы: «Работа проходит сменами по девять месяцев. Предлагаемый график на смене – 5 на 2. После каждой смены необходимо менять город проживания. Города могут повторяться, но только через три года после прошлого посещения. Нерабочие три месяца будут оплачиваться четвертью зарплаты за месяц». Заголовок с текстом: «Оплата» был вынесен отдельно. Под ним виднелась короткая строчка «Минимальная заработная плата в год составляет 500.000 долларов», – Том уставился на эту строчку, непонимающим взглядом. На предыдущей работе его теоретическая зарплата за год еле доходила до семидесяти тысяч. 

– Пять... 

– Прошу, ведите себя тихо, Мистер Браун, не надо поднимать шум. Что вы думаете? – перебил его добродушный на вид толстяк напротив. 

– Эээ... над чем я буду работать? Это законно? 

– Ха-ха, законно ли это? Зависит от точки зрения. Для государства – да, для морали – нет. Вы будете работать по специальности, но с чуть отличающимися вещами. Вот, – доктор протянул ему красную папку. – Ознакомьтесь. 

Вновь посмотрев на него с недоумением, Том открыл папку, отложив прошлую. В ней также был только один лист. «Создание людей с неординарными способностями».

– То есть эксперименты на или с лю...

– Да, вы все правильно поняли, – снова перебил его Нуотс. 

– Ваш кофе и вода, мистер, – вмешалась официантка.

– Спасибо... спасибо, – растерянно проговорил Том. 

– Вам что-нибудь принести? – обратилась она к доктору.

– Нет, спасибо. Мы уже почти закончили, – добро улыбаясь, отвечал он. 

«Боже, во что я ввязался? Как они нашли меня и почему выбрали меня? С другой стороны, это легальная работа с очень неплохой зарплатой. Да и условия интересны. Это позволит мне пожить в стольких местах, ведь я не буду привязан к конкретному городу. Но все же, это же люди... это некорректно. С другой стороны, при создании с нуля, можно ли это будет считать человеком? А если это будет не снуля? Я... нет. Нет. Это неправильно. И я не буду...»

– Так что? – прервал размышления его собеседник. 

– Боюсь, что мне придется отказаться. Я не готов к такому. 

– Очень жаль. Очень. Прежде, чем я уйду, посмотрите еще одну вещь, прошу, – Нуотс протянул ему последнюю папку фиолетового цвета. Том вновь принял ее и, открыв ее, увидел лишь две строчки: 

\begin{center}«Совершенно секретно.
Все гражданские, узнавшие про это, подлежат уничтожению.»\end{center}


Он медленно поднял глаза на Нуотса и спросил: 

– То есть, у меня с самого начала не было выбора?  

– Нуу... боюсь, что именно так. Что вы думаете теперь? Может дополнительные условия подтолкнули вас к верному выбору? – сказал толстяк с хитрым прищуром, пожимая плечами. 

Том тяжко вздохнул. Нельзя сказать, что он был абсолютно против данного мероприятия. Но что-то его тревожило. Он чувствовал, что совершает преступление против чего-то большего, чем государство. «С другой стороны, – думал он. – С такими деньгами я мог бы помочь другим людям. Таким образом, мне бы становилось легче, наверное. Я смог бы помочь Майклу бросить все его нелегальные дела. По крайней мере, умирать я точно не хочу».

– Ладно! Хорошо! Договорились, – наконец сказал Том, все еще терзаемый слабым чувством вины. 

– Вот и замечательно, вот и хорошо. Вам хватит недели на улаживание всех дел? Если да, то я пришлю вам машину в следующую пятницу вечером. Все, что вы сможете купить на новом месте, оставляйте. Можете собрать до трех чемоданов вещей, – сказал довольный Нуотс

– У меня тут не так много дел, я буду готов завтра вечером, – отвечал Том, уже допивая воду. 

– Отлично-отлично. Ну что, пожмем руки и пойдем? – доктор протянул руку и Том ее пожал. 

\begin{center}***\end{center}

– Стой-стой-стой!! То есть ты пропал, начав работать на секретную государственную организацию по созданию супергероев?! – прервал рассказ Майкл

– Кхм... ну если так подумать, то, в принципе, да, – отвечал перебитый Том.

– Обалдеть...

– Мне продолжать? – спросил сидящий на стуле

– Да, конечно. Я будто в глупой книжке сейчас, честно говоря. 
\clearpage
{\begingroup
\ThisCenterWallPaper{1.0}{после 5 части.jpg}
\noindent
\endgroup}
\cleardoublepage
