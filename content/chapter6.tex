\chapter{Последние сборы}
\lettrine{Н}{}а часах уже полночь. Человек идет по практически пустой улице. На нем надет идеально подходящий ему костюм вкупе с темно-коричневыми балморалами и белой, словно свежевыпавший снег, рубашкой. Весь его вид, был словно одним большим оммажем на классический вид английского джентльмена второй половины двадцатого века. Том шел обратно в квартиру, но уже длинным путем. Ничто больше не тревожило его. Он ощущал что-то, что он никак не мог охарактеризовать. «Может, это счастье? Но что я тогда чувствовал раньше? Хм... нет. Это, скорее,некое благоговение. Но пред чем? Наверное, пред будущем», – думал Том, все идя по пустынному ночному городу. Он наслаждался каждым шагом, каждым вдохом. Ночной воздух для него всегда был чем-то особенным. Была в нем какая-то особая магия, особое неуловимое ощущение, что невозможно было почувствовать днем. «Конечно, очевидно, что так ощущается лишь оттого, что солнце сейчас не так сильно нагревает различные материалы с последующим выделением специфических запахов, что различные отходы замедляют свое разложение, что у растений спадает активность опыления. Н-да. Чем больше ты знаешь, тем меньше чудес остается на земле, – растерянно смотря на яркие звезды, думал Том. – А ведь когда-то все, что я видел, было для меня таким загадочным и необычным. Даже обыденная сейчас вещь казалась чем-то невероятным. Хотя... сейчас просто сменились масштабы. Да и поразительного не стало меньше, просто я привык к этому чувству. Так что надо держать это в уме и вспоминать, когда следует. Да».

Том вернулся во все еще блестяще чистую квартиру. Судя по всему, это была последняя ночь, которую он проведет в ней. Но он не сожалел об этом – ничто не связывало его с ней. Кроме облегчения и легкой усталости после покупок Том ничего не ощущал. Сняв всю одежду, он улегся в кровать. Сон долго не приходил к нему, поэтому он смотрел в потолок, думая о завтрашнем дне. 

Легкие лучи солнца падали на лицо Тома, и он начал медленно подниматься. Человек оделся и, пританцовывая, отправился на кухню: «Утренний кофе еще никому не мешал», – думал он. Уже допивая кофе, Тома посетили мысли: «А что, собственно, мне делать все оставшееся время? Так. Надо вернуть ключи хозяйке, но, наверное, я смогу просто оставить их в почтовом ящике, и она их оттуда заберет. Да, пожалуй, так, но стоит позвонить – уточнить. Хм. Возможно, стоит позвонить Майклу, хоть в прошлый раз я так и не дозвонился до него. Жив ли он еще? Ха! Глупый вопрос, это изворотливое животное выберется из любой ситуации», – подумал Том, усмехнувшись себе под нос. Допив кофе, персона поднялась и пошла к телефону, набрала хозяйку. Разговор был весьма коротким, с минимальным количеством бессмысленной болтовни. Они сошлись на том, что он оставит ключи в почтовом ящике.

– Как я и думал, – сказал вслух Том, положив трубку. 

Он занес руку для набора еще одного номера. Палец уже опустился на цифру семь, но не нажал ее. Том не был уверен, что ему стоит звонить своему брату. Никто не мог гарантировать, что он возьмет трубку. А если Майкл не ответит, то это лишь даст старшему брату повод для волнений. Перст медленно удалялся от телефона, а вторая рука вешала трубку. Он так и не позвонил ему. 

Том перетащил чемодан ко входу в надежде, что во время этого процесса он вспомнит, что ему нужно сделать. Но этого не случилось, и оставшиеся часы он просто слонялся по квартире без видимой цели. Наконец, солнце начало подходить к горизонту, окрашивая облака в золотисто-розовый цвет, заливая улицы теплыми оттенками, что повышают всеобщий дух и настроение. В квартире зазвонил телефон. Том незамедлительно ответил и услышал лишь одну фразу: «Том Браун, здравствуйте, машина уже у дома».  Взяв чемодан, Том вышел из квартиры. Спускаясь по лестнице, преодолевая пролет за пролетом, он чувствовал, словно что-то тянет его назад. Но это ощущение было ничтожно мало в сравнении с тем, что вело его вперед. Он вышел, глубоко вдохнул и сел в черную машину, стоящую пред самым подъездом. 
В машине он увидел уже знакомое лицо доктора Нуотса. Волосы его были немного влажными от пота – в машине было довольно душно. От водителя их отделяла темная стеклянная перегородка. 

– Здравствуйте, Том! Рад видеть вас, – сказал Нуотс, проводя платком по лбу. – Ух... что-то больно душно здесь...

–  Здравствуйте, – сказал Том, усаживаясь. 

– Фух... – толстяк постучал по перегородке, и она опустилась. – Послушайте, а нельзя ли включить кондиционер? 

– Да, как угодно, – ответила грузная темная фигура, сидящая за рулем, и отправила свою руку к приборной панели автомобиля. Стекло вновь отделило доктора, Тома и чемодан, который почему-то был не в багажнике, а в руках у преспокойного человека. 

– Хох. Так намного лучше. Вы, наверное, хотели бы знать, почему выбрали именно ва... – начал говорить Нуотс, но его прервал слушатель.

– Нет. Я не хотел бы это узнать. Это меня уже не волнует, – твердым, как сталь, голосом сказал Том, смотря прямо в маленькие глаза доктора. – Я предпочту узнать, как и куда мы будем добираться, – он никогда не доверял полным людям. Нуотс не стал для него исключением. 

– Как пожелаете, – его собеседник поднял руки, будто бы сдаваясь. – Сейчас мы доедем до аэропорта, там нас будет ждать самолет прямо до пункта назначения. А летим мы на крайний север нашей необъятной. Там-то вы и будете проводить по девять месяцев в году, – все так же добродушно приятно говорил доктор. Том не знал, что ему еще можно спросить. Он сможет узнать все на месте, а сейчас стоит просто расслабиться и «плыть по течению».

– Хорошо... хорошо. А музыку можно поставить? «42.ФМ», если вы не против. 

– Конечно, – Нуотс вновь постучался к силуэту, а его собеседник закрыл глаза и откинулся на спинку сиденья. Он не спал, но мыслительный процесс настолько захватил его, что окружающая действительность уже не была для него принципиальна. Лишь приезд в аэропорт вырвал его из этого состояния, да и то – ненадолго. Только на время пересадки из машины в роскошный личный самолет. В нем уже не было радио, и пришлось провести весь полет в гудящей тишине. Никто не нарушал ее. Звездное небо виднелось в иллюминатор, а луну Том так и не увидел. 
\clearpage
{\begingroup
\ThisCenterWallPaper{1.0}{после 6 части.jpg}
\noindent
\endgroup}
\cleardoublepage
