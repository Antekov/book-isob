Они прошли сквозь дверной проем и оказались в обширнойкомнате. Она была выполнена в строгом минимализме с любовью к пространству. Чуть глубже виднелся огромный массивный стол, судя по всему, сделанный изкакой-то благородной древесины. Крышка была частично покрыта нежно-зеленойтканью, подобной бювару, но вделанной в стол. На правом краю стоял графин сводой и высокий гладкий стакан с тяжелым дном. Справа от стола у стены стоял секретер, выполненный из той же древесины. На нем стояла лишь «пьющая птичка» с ярко красной головой, белой шляпкой иголубым хвостом. Как раз рядом с ней стояла маленькая пиала с небольшимколичеством воды в ней. Слева же от стола практически пустая стена. По центрустены была расположена дверь, а по бокам от нее висели две картины с несколькосюрреалистичными мотивами, выполненными в очень ярких цветах. Это придавалокомнате своеобразный шарм, позволяло ей быть одновременно безумно строгой, носохранять некую пелену уюта, что окутывал входящего, словно теплый плед. А застолом себе нашло место монструозное панорамное окно. Глядя в него, Том виделчудесный пейзаж, состоящий наполовину из сплошной стены зимнего леса, а другуюполовину захватило серо-белое небо. Тому казалось, что это не может быть егокабинет. Что, возможно, произошла какая-та ошибка. Но с другой стороны, емубыло безумно приятно даже находится здесь.

— Так, ну, в принципе, обустраиваетесь. Я вернусь к вам чутьпозже, — мило улыбнувшись сказала Карен, уже закрывая дверь. Ошеломленный Томдаже не успел сказать ей что-нибудь в прощание.

— Кхм, — произнес он уже в пустой комнате. — Ну и ладно.

Еще пару мгновений наслаждаясь интерьером комнаты, оннаконец-то вспомнил про боковое помещение, отделенную от него лишь не самойтолстой деревянной поверхностью. К счастью, эта дверь не требовала отдельного, да хоть какого-нибудь ключа, поэтому Том сразу же смог зайти в комнату. Этооказалась спальня. Большую часть пространства занимала двуспальная кровать. Сначала это немного смутило Тома, но после он остался весьма доволен. Спать натаком огромном и мягком матрасе было исключительно благодатно. В остальном же, комната была также аккуратнаи немногословна в отношении вещей. Слева от кровати стояла тумбочка, со стоящейна ней лампой. У левой стены: платяной шкаф, некий комод и, висящий на стене, телефон. «Они все делают из этого дерева?» — думал Том, оглядывая комнату. Взглянув направо, он увидел дверь, которая, скорее всего, вела в уборную, и свой чемодан, который неведомым образом достигего «квартиры».

______________________________________________________

Том остановил свое повествование, увидев, что Майкл уснул.

— Хох… совсем утомился, видимо. Впрочем, это неудивительно. Он довольно сильно пострадал, — шепотом сказал старший из братьев и встал. –Надо бы проведать второго «пострадавшего», — добавил он, выходя в несколькоосточертевший ему коридор.

Том лишь недавнопонял, насколько ему приелся этот вид. Единственное, что спасало, так этоотсутствие времени на раздумья о нем в связи с огромным количеством работы. Дажесейчас онподумал об этом лишь мгновение, вернувшись к мыслям об исследуемом объекте №53.Спустя несколько лет работы его труды увенчалисьуспехом. Он дошел до сделанной по шаблону двери, но с другим номером и, приложив ключ-карту, открыл ее. Том оказался в помещении с немного приглушеннымсветом. Его окружали различные считывающие устройства, компьютеры, а впередибыло видно одностороннее стекло. За ним, под лампами дневного света, на диванесидел №53 в белой больничной пижаме с нехитрым узором из кошачьих мордочек.

— Карен, состояние объекта, — требовательно сказал Том. Запределами лаборатории они, может, и были друзьями, но здесь без строгой субординацииобойтись было невозможно.

— Состояниестабильное, тело полностью восстановилось после внешнего воздействия, никакихотклонений замечено не было, — мгновенно отвечала Морган, смотря на мониторы изаписывая ключевую информацию в свою тетрадь под новой датой.

— Хорошо-хорошо. Боб, я собираюсь пойти к нему. Где таблетки? — Том перевел взгляд в другой конецкомнаты и увидел уже идущего к нему мужчину с баночкой в руке.

– Вот, доктор Браун, возьмите.

— Спасибо, — Томоткрутил крышку и проглотил небольшую желтую капсулу. — Так, Гэри, ты составилотчет о побеге? — ему нужно было подождать пять минут, пока подействуеттаблетка.

— Да, доктор Браун. Вследствиерасследования было выявлено, что злоумышленник проник в лабораторию и смогвыкрасть объект с применением силы, из-за чего и погиб от кровоизлияния в мозгуже на выходе с территории. Объект же бежал в тогда еще неизвестном направленииво время бури, по причине которой и был утерян его сигнал… — высокий блондинхотел продолжить доклад, но Том, посмотрев на часы, прервал его.

— Хорошо, достаточно. Охранники? — спросил он, подходя к двери сбоку от одностороннего стекла.

— Были разжалованы и былилишены памяти обо всем, что было связано с этим местом, — четко проговорилГэри. — Назначены новые лица.

— Жестоко, носправедливо. Карен, открывай, — Том уже держал руку на двери, готовый заходить. Легкая женская рука прошлась по нескольким кнопкам, подтверждая вход и дверьотворилась.

Том зашел в комнату, взял стоящий напротив двери стул и поставил его спинкой к дивану. Сел на него, сложив руки на спинке под внимательным взглядом молчащего зрителя.

— Привет, Исоб, –дружелюбно произнес он.

 — Привет, друг! Тыне поверишь, что со мной произошло! — взволнованно произносил бледный человек, глядя прямо в глаза собеседнику.

— О, да ты менязаинтересовал, дружище. Расскажешь? Я вот, — Том достал небольшой блокнот иручку из внутреннего кармана халата. — Даже запишу, если ты не против, конечно, — сидящий на стуле посмотрел в глаза собеседника. Тот сразу же отвел взгляд, почувствовав превосходящую силу.

— Сначала ко мнепришел какой-то человек. Он был не другом, — Исоб прервался, задумчиво уставившись в потолок. — Он дал мне какую-тостранную одежду и сказал переодеваться. Мне он не понравился. Он все время меняторопил, а одежда была какая-то неудобная, — Том смотрел то на рассказчика, тов свой блокнот, в котором он якобы делал записи. Разумеется, полнаякартина всех событий была ему уже известна. Он даже не нуждался в информации отГэри, но в каком-то смысле это было частью его работы: все время держать всех втонусе, не давать расслабится во время работы.

— Потом мы долго шлипо длинным коридорам, как будто на ту игру. Ну, на которую я с вами ходил, –продолжал говорить №53. — Мы шли молча. Потом мы вышли на улицу, и он наделтакую странную штуку. Это было что-то вроде верха от моей пижамы, но толстое. Иновый друг тоже ее носил. А потом я видел такое, — Исоб вновь посмотрел наТома своими расширившимися от удивления глазами. — Я видел такую белую штуку. Она забавно щекотала кожу, а в нос словно лезло что-то. Это так странно, друг, — рассказчик очень активно жестикулировал руками, но движения его были довольнохаотичны. Фактически невозможно было найти в них связь с повествованием. –Когда мы выходили в такую широкую дверь, я упал. За это он тихо накричал наменя. Пока мы шли по этой странной штуке, о-о, она так забавно касается ног, друг. Пока мы шли, я немного останавливался и за это он тоже тихо кричал наменя. Он плохой, — уверенно произнес человек в пижаме. — А потом мыпроходили через еще какую-то штуку. Она была серой и в дырочках. Он ее отогнули сказал проходить. Я сразу справился, друг!

— Молодец, — умереннопохвальным тоном сказал Том, продолжая притворно записывать что-то.

— Потом мы подошли ктакой штуке, хм… — №53 задумался. — Как большая коробка, с отверстиями подтакие круглые вещи. Колеса! Они называются колеса! Но когда мы подошли к этойсерой штуке, он упал на землю, а я услышал очень громкий звук оттуда, откуда мыпришли! Я так испугался, друг! И я вспомнил твои слова: «всегда двигайсявперед». Поэтому я побежал прямо. Я бежал довольно долго, но потом я устал, и бежать становилосьвсе сложнее. Но я вспоминал твои слова и шел дальше, пока не упал. Я сначалахотел подняться и продолжить идти, но не смог. Я повернул голову и увидел такойяркий свет. Я очень испугался, ведь вспомнил, как мне делали ту о… — он насекунду остановился и закрыл глаза. — Ту операцию, после которой у меня сильно болело здесь, — Исоб показал на свой бок.

Том кивал головой, думая, что в будущем стоит серьезно взяться за образование №53, ведь этобанально сложно слушать.

— Я даже заснул, прямо как перед операцией! Но потом я проснулся в такой странной комнате. В нейбыл такая штука. Она была похожа на теплый, даже горячий, свет. Майкл сказал, что это огонь. Еще там был такой необычный запах. Мне почему-то так сложно былоговорить. Словно язык путался. Когда я проснулся, я увидел нового друга. Сначалая попросил у него воды, а только потом уже начал знакомиться, — Исоб посмотрелна Тома, словно ожидая некоторого выговора, но тот лишь отвел взгляд отблокнота и посмотрел на своего собеседника. — Но потом я делал все так, как выменя учили. Я познакомился с ним: его звали Майкл. Потом я долго смотрел наогонь: он рисовал столько различных картинок, друг! Но тут другой друг досталэти белые штуки, хм, сигареты, прямо как на той игре с людьми. И я сделал все вточности, как вы меня учили. Только в конце я выбросил ее не в ведро, а бросилв огонь, но это же не страшно?

— Конечно, нестрашно, можешь об этом не волноваться, — кивнул Том.

— А потом он начал задаватьочень много вопросов, и я очень испугался. Поэтому я снова заснул. Во сне яначал видеть цвета. Мои глаза были закрыты, но я видел красный, белый, синийцвета. Что это было?

— Это был сон, Исоб. Ты видел сон, — на этот раз Томуже действительно сделал пометку в блокноте. До этого ни один из объектов неимел сновидений. «Конечно, это весьма простой сон, но все же…» — думал Том.

— Ну вот ивсе. А потом я уже проснулся здесь, — он оглянулся по сторонам. — А когда япопаду в свою комнату? — спросил №53, уставившись на своего собеседника.

— Хм… — Том задумчивопосмотрел в ответ. — Сейчас я ненадолго отойду и, вернувшись, скажу, — Томподнялся со стула и, взяв его в руки, пошел к выходу, сопровождаемый взоромИсоба.

Выйдя из второйполовины комнаты, он сразу же поинтересовался. проснулся ли Майкл. На чтополучил отрицательный ответ. «Тогда я, пожалуй, схожу к доктору Нуотсу. Надо быуточнить поповоду перемещения №53», — поразмыслил он. И вновь длинный коридор, но на этотраз путь был совсем коротким: кабинет доктора находился всего через парудверей. Том постучался и, услышав ответное «входите, открыто», нажал на дверь. Он оказался в довольно похожем на его собственный кабинет. Еще когда он впервый раз пришел сюда, он подумал: «Почему они такие одинаковые?». Лишькартины на стенах отличались. Если у Тома были более сюрреалистичные мотивы, тоу его коллеги преобладали классические произведения эпохи Возрождения.

– И сноваздравствуйте, доктор Нуотс! — сказал Том, заходя.

— Приветствую. Какваш разговор с братом? — дружелюбно принимал гостя толстяк. Том слышал, что тотнесколько раз пытался сесть на диету и заняться спортом, но безуспешно.

— Ну, он заснул. Видимо рассказ мой был недостаточно интересен, — с некоторой иронией произнесБраун.

— Оу, впрочем, онсильно пострадал, так что это не удивительно. Вы по делу?

— Да. Есть какие-либопричины не переводить №53 обратно в его… — Том запнулся. — Комнату?

— Нет, можете спокойно вести его туда, –утвердительно сказал Нуотс.

— Хорошо, доктор. Тогда я немедленно этим и займусь, — гость уже собирался выходить, нотут его остановили.

— Том Браун… Том… Подождите, — чувствовалось, что человеком за столом пытается подобрать нужнуюинтонацию, но ни одна из них его не устраивала. — Мы не можем оставить Майклаеще на некоторое время. Это не входило в наш бюджет и, — он вздохнул. — Мынемедленно отправим его обратно в хижину. Там уже навели порядок. Мы сотрем емупамять обо всем произошедшем и внедрим ложные воспоминания о происхождении ран, — четко проговорил остаток предложения Нуотс.

— Ох, — Тому былонесколько неприятно, что это происходит. Он хотел бы дорассказать историю, да и он так давно не видел Майкла. Атеперь он даже вряд ли сможет с ним встретится. Ведь Майкл сменит имя, город, жизнь, а сам Том не имеет постоянного адреса и телефона. Поэтому все попытки состороны Майкла тоже будут бессмысленны. Неприятное чувство росло в груди ипостепенно начинало давить на горло. — Значит. Будет так. Досвидания, доктор.

Том вышел из кабинета, оставив все свои переживания о случившимся там же. Сейчас у него есть работа, не оставляющая времени на эти раздумья.

______________________________________________________

— Хей, Исоб, явернулся, — весело произнес Том. — У тебя же тут нет вещей, так? Тогда пойдемобратно в твою комнату. Только возьми эти тапочки, полы ведь холодные.

— Ура! — обрадовавшийся №53 принял обувь, и, надев ее, встал, уже готовыйидти.

Они вышли в коридор иотправились в путь. Том за годы работы уже выучил все дороги и развики. Теперьполусферы стали для него лишь элементом декора. Они двигались вперед. ДокторБраун не совсем понимал, почему Исоба сразу не вернули в его комнату, апоместили в почти полностью пустую, да еще и в другом конце здания. Мерныетяжелые шаги и небольшое шарканье тапочек врывались в тишь каждого коридора, предупреждая о приближении людей растения и недвижимые двери. Внезапно шарканьепрекратилось, а за ним через несколько секунд и тяжелые шаги.

Том обернулся и увидел, что №53 остановился, схватившись заголову. Мгновение спустя по коридору прокатился громкий стон.

— Исоб?!

— Д-д-друг… мне… мне плохо… я… что-то чувствую, я слышукрики. Мне страшно. Мне больно.

«Что… что происходит? — мысли хаотичноносились в голове доктора. — Черт. Черт. ЧЕРТ!» — Том понял, где они находятся. Крик Исоба эхом разнесся по пустому пространству. Прямо за этой дверью, напротив которой стояло два человека, находилось своеобразное «кладбище». Здесьбыли уничтожены все предыдущие объекты. Медленно растоплены в кислотных ваннах. Конечно, они были уже мертвы. Но это было самой очевидной идеей для Тома. Страдания каждого объекта, вся их боль, были скоплены и запечатаны в этой стене, двери, помещении. И прямо сейчас на глазах у доктора Брауна стоял тот, кто открылэтот ящик Пандоры. Тот, кто заберет все содержимое себе, пусть и против своейволи.

Том бросился к кричащему и, схватив его, понес как можнодальше отсюда. Он бежал, оглушенный паникой, разрывающей его изнутри, но лицоего не выражало даже толики этого чувства. Наконец, истратив все силы, ониостановились. Том положил №53 на холодный пол и тут страшное осознание пронзилоего разум. Исоб был мертв. Он умер. Том сел на пол, не зная, что ему теперь делать. Его труд был уничтожен им же. Конечно, остались все записи, все образцы. Но он знал, что ему уже не предстоитработать с ними. Это была халатность. Огромная ошибка. Уже остывающее тело, которое было покинуто разумом, лежало перед ним. Мысли метались в его голове, словно ягоды в блендере. «Это конец. Меня отстранят, сотрут память, отправяткуда-то, неизвестно куда. Но уж точно не на курорт. Скорее наоборот. Этоконец». Том взялся за голову и сжал ее руками с двух сторон, словно желаяраздавить ее своими руками, подобно прессу. Не преуспев в этом, он огляделся. Ослепленный паникой, он пришел прямо к своему кабинету. И тут между мыслью отом, что это конец, затесалась новая. Спасительная идея. «Это единственное, чтоможно сделать» — думал он, открывая дверь в свой кабинет. Здесь, в этой самойкомнате, лежал ключ. Ключ, который онпрямо сейчас использует для выхода из этой ситуации. Выдвинув первый ящик столаи вытащив оттуда уже бесполезные папки, его встретил он. Аккуратный «дамский»пистолет, с небольшой гравировкой в виде лозы на стволе.

– Так будет лучше, — прозвучало в пустой комнате передследующим, теперь уже оглушающим, звуком.

______________________________________________________

Майкл проснулся надиване перед камином. Ему что-то приснилось, но уже спустя мгновение он никакне мог вспомнить, что же было в этом сне. На него также, как и вчера, смотрелавеличественная голова кабана, а за окном бушевала метель. Раны, полученные приперестрелке с участниками картеля, неприятно напоминали о себе. «Видимо, янесколько раз терял сознание уже в хижине, — думал он. — Еще и голова трещит». Повернувшись на бок, он вновь увидел книжные полки, наполненные самыми различнымиавторами и жанрами. «Что же, впереди долгая неделя», — он поднялся и подошелпоближе к корешкам, скрывающим за собою захватывающие истории.

Оказалось, что неделя– это не так уж и долго. Погрязнув в готовке, уборке, колке дров, чтении, размышлениях перед огнем, Майкл даже немного огорчился, получив сообщение, чтовсе готово. Странная тоска окутала его, когда он покидал хижину. Ему казалось, что за это время он прошел какое-то удивительное путешествие, а теперь же онозакончилось, так и не дав ответов на все его вопросы. Одинокая машина двигаласьпо шоссе, залитому ярким солнечным светом. На дорогу из окна выпало нескольконепочатых пачек сигарет.

— Что бы Том сказалмне сейчас? — прозвучало в салоне, под аккомпанемент из медленной музыки, чтозатягивает слушателя в раздумья все глубже и глубже.

Они прошли сквозь дверной проем и оказались в обширной комнате. Она была выполнена в строгом минимализме с любовью к пространству. Чуть глубже виднелся огромный массивный стол, судя по всему, сделанный из какой-то благородной древесины. Крышка была частично покрыта нежно-зеленой тканью, подобной бювару, но вделанной в стол. На правом краю стоял графин с водой и высокий гладкий стакан с тяжелым дном. Справа от стола у стены стоял секретер, выполненный из той же древесины. На нем стояла лишь «пьющая птичка» с ярко красной головой, белой шляпкой и голубым хвостом. Как раз рядом с ней стояла маленькая пиала с небольшим количеством воды в ней. Слева же от стола практически пустая стена. По центру стены была расположена дверь, а по бокам от нее висели две картины с несколько сюрреалистичными мотивами, выполненными в очень ярких цветах. Это придавало комнате своеобразный шарм, позволяло ей быть одновременно безумно строгой, но сохранять некую пелену уюта, что окутывал входящего, словно теплый плед. А за столом себе нашло место монструозное панорамное окно. Глядя в него, Том видел чудесный пейзаж, состоящий наполовину из сплошной стены зимнего леса, а другую половину захватило серо-белое небо. Тому казалось, что это не может быть его кабинет. Что, возможно, произошла какая-та ошибка. Но с другой стороны, ему было безумно приятно даже находится здесь.

— Так, ну, в принципе, обустраиваетесь. Я вернусь к вам чуть позже, — мило улыбнувшись сказала Карен, уже закрывая дверь. Ошеломленный Том даже не успел сказать ей что-нибудь в прощание.

— Кхм, — произнес он уже в пустой комнате. — Ну и ладно.

Еще пару мгновений наслаждаясь интерьером комнаты, он наконец-то вспомнил про боковое помещение, отделенную от него лишь не самой толстой деревянной поверхностью. К счастью, эта дверь не требовала отдельного, да хоть какого-нибудь ключа, поэтому Том сразу же смог зайти в комнату. Это оказалась спальня. Большую часть пространства занимала двуспальная кровать. Сначала это немного смутило Тома, но после он остался весьма доволен. Спать на таком огромном и мягком матрасе было исключительно благодатно. В остальном же, комната была также аккуратна и немногословна в отношении вещей. Слева от кровати стояла тумбочка, со стоящей на ней лампой. У левой стены: платяной шкаф, некий комод и, висящий на стене, телефон. «Они все делают из этого дерева?» — думал Том, оглядывая комнату. Взглянув направо, он увидел дверь, которая, скорее всего, вела в уборную, и свой чемодан, который неведомым образом достиг его «квартиры».

______________________________________________________

Том остановил свое повествование, увидев, что Майкл уснул.

— Хох… совсем утомился, видимо. Впрочем, это неудивительно. Он довольно сильно пострадал, — шепотом сказал старший из братьев и встал. — Надо бы проведать второго «пострадавшего», — добавил он, выходя в несколько осточертевший ему коридор.

Том лишь недавно понял, насколько ему приелся этот вид. Единственное, что спасало, так это отсутствие времени на раздумья о нем в связи с огромным количеством работы. Даже сейчас он подумал об этом лишь мгновение, вернувшись к мыслям об исследуемом объекте №53. Спустя несколько лет работы его труды увенчались успехом. Он дошел до сделанной по шаблону двери, но с другим номером и, приложив ключ-карту, открыл ее. Том оказался в помещении с немного приглушенным светом. Его окружали различные считывающие устройства, компьютеры, а впереди было видно одностороннее стекло. За ним, под лампами дневного света, на диване сидел №53 в белой больничной пижаме с нехитрым узором из кошачьих мордочек.

— Карен, состояние объекта, — требовательно сказал Том. За пределами лаборатории они, может, и были друзьями, но здесь без строгой субординации обойтись было невозможно.

— Состояние стабильное, тело полностью восстановилось после внешнего воздействия, никаких отклонений замечено не было, — мгновенно отвечала Морган, смотря на мониторы и записывая ключевую информацию в свою тетрадь под новой датой.

— Хорошо-хорошо. Боб, я собираюсь пойти к нему. Где таблетки? — Том перевел взгляд в другой конец комнаты и увидел уже идущего к нему мужчину с баночкой в руке.

– Вот, доктор Браун, возьмите.

— Спасибо, — Том открутил крышку и проглотил небольшую желтую капсулу. — Так, Гэри, ты составил отчет о побеге? — ему нужно было подождать пять минут, пока подействует таблетка.

— Да, доктор Браун. Вследствие расследования было выявлено, что злоумышленник проник в лабораторию и смог выкрасть объект с применением силы, из-за чего и погиб от кровоизлияния в мозг уже на выходе с территории. Объект же бежал в тогда еще неизвестном направлении во время бури, по причине которой и был утерян его сигнал… — высокий блондин хотел продолжить доклад, но Том, посмотрев на часы, прервал его.

— Хорошо, достаточно. Охранники? — спросил он, подходя к двери сбоку от одностороннего стекла.

— Были разжалованы и были лишены памяти обо всем, что было связано с этим местом, — четко проговорил Гэри. — Назначены новые лица.

— Жестоко, но справедливо. Карен, открывай, — Том уже держал руку на двери, готовый заходить. Легкая женская рука прошлась по нескольким кнопкам, подтверждая вход и дверь отворилась.

Том зашел в комнату, взял стоящий напротив двери стул и поставил его спинкой к дивану. Сел на него, сложив руки на спинке под внимательным взглядом молчащего зрителя.

— Привет, Исоб, — дружелюбно произнес он.

 — Привет, друг! Ты не поверишь, что со мной произошло! — взволнованно произносил бледный человек, глядя прямо в глаза собеседнику.

— О, да ты меня заинтересовал, дружище. Расскажешь? Я вот, — Том достал небольшой блокнот и ручку из внутреннего кармана халата. — Даже запишу, если ты не против, конечно, — сидящий на стуле посмотрел в глаза собеседника. Тот сразу же отвел взгляд, почувствовав превосходящую силу.

— Сначала ко мне пришел какой-то человек. Он был не другом, — Исоб прервался, задумчиво уставившись в потолок. — Он дал мне какую-то странную одежду и сказал переодеваться. Мне он не понравился. Он все время меня торопил, а одежда была какая-то неудобная, — Том смотрел то на рассказчика, то в свой блокнот, в котором он якобы делал записи. Разумеется, полная картина всех событий была ему уже известна. Он даже не нуждался в информации от Гэри, но в каком-то смысле это было частью его работы: все время держать всех в тонусе, не давать расслабится во время работы.

— Потом мы долго шли по длинным коридорам, как будто на ту игру. Ну, на которую я с вами ходил, — продолжал говорить №53. — Мы шли молча. Потом мы вышли на улицу, и он надел такую странную штуку. Это было что-то вроде верха от моей пижамы, но толстое. И новый друг тоже ее носил. А потом я видел такое, — Исоб вновь посмотрел на Тома своими расширившимися от удивления глазами. — Я видел такую белую штуку. Она забавно щекотала кожу, а в нос словно лезло что-то. Это так странно, друг, — рассказчик очень активно жестикулировал руками, но движения его были довольно хаотичны. Фактически невозможно было найти в них связь с повествованием. — Когда мы выходили в такую широкую дверь, я упал. За это он тихо накричал на меня. Пока мы шли по этой странной штуке, о-о, она так забавно касается ног, друг. Пока мы шли, я немного останавливался и за это он тоже тихо кричал на меня. Он плохой, — уверенно произнес человек в пижаме. — А потом мы проходили через еще какую-то штуку. Она была серой и в дырочках. Он ее отогнул и сказал проходить. Я сразу справился, друг!

— Молодец, — умеренно похвальным тоном сказал Том, продолжая притворно записывать что-то.

— Потом мы подошли к такой штуке, хм… — №53 задумался. — Как большая коробка, с отверстиями под такие круглые вещи. Колеса! Они называются колеса! Но когда мы подошли к этой серой штуке, он упал на землю, а я услышал очень громкий звук оттуда, откуда мы пришли! Я так испугался, друг! И я вспомнил твои слова: «всегда двигайся вперед». Поэтому я побежал прямо. Я бежал довольно долго, но потом я устал, и бежать становилось все сложнее. Но я вспоминал твои слова и шел дальше, пока не упал. Я сначала хотел подняться и продолжить идти, но не смог. Я повернул голову и увидел такой яркий свет. Я очень испугался, ведь вспомнил, как мне делали ту о… — он на секунду остановился и закрыл глаза. — Ту операцию, после которой у меня сильно болело здесь, — Исоб показал на свой бок.

Том кивал головой, думая, что в будущем стоит серьезно взяться за образование №53, ведь это банально сложно слушать.

— Я даже заснул, прямо как перед операцией! Но потом я проснулся в такой странной комнате. В ней был такая штука. Она была похожа на теплый, даже горячий, свет. Майкл сказал, что это огонь. Еще там был такой необычный запах. Мне почему-то так сложно было говорить. Словно язык путался. Когда я проснулся, я увидел нового друга. Сначала я попросил у него воды, а только потом уже начал знакомиться, — Исоб посмотрел на Тома, словно ожидая некоторого выговора, но тот лишь отвел взгляд от блокнота и посмотрел на своего собеседника. — Но потом я делал все так, как вы меня учили. Я познакомился с ним: его звали Майкл. Потом я долго смотрел на огонь: он рисовал столько различных картинок, друг! Но тут другой друг достал эти белые штуки, хм, сигареты, прямо как на той игре с людьми. И я сделал все в точности, как вы меня учили. Только в конце я выбросил ее не в ведро, а бросил в огонь, но это же не страшно?

— Конечно, не страшно, можешь об этом не волноваться, — кивнул Том.

— А потом он начал задавать очень много вопросов, и я очень испугался. Поэтому я снова заснул. Во сне я начал видеть цвета. Мои глаза были закрыты, но я видел красный, белый, синий цвета. Что это было?

— Это был сон, Исоб. Ты видел сон, — на этот раз Том уже действительно сделал пометку в блокноте. До этого ни один из объектов не имел сновидений. «Конечно, это весьма простой сон, но все же…» — думал Том.

— Ну вот и все. А потом я уже проснулся здесь, — он оглянулся по сторонам. — А когда я попаду в свою комнату? — спросил №53, уставившись на своего собеседника.

— Хм… — Том задумчиво посмотрел в ответ. — Сейчас я ненадолго отойду и, вернувшись, скажу, — Том поднялся со стула и, взяв его в руки, пошел к выходу, сопровождаемый взором Исоба.

Выйдя из второй половины комнаты, он сразу же поинтересовался. проснулся ли Майкл. На что получил отрицательный ответ. «Тогда я, пожалуй, схожу к доктору Нуотсу. Надо бы уточнить по поводу перемещения №53», — поразмыслил он. И вновь длинный коридор, но на этот раз путь был совсем коротким: кабинет доктора находился всего через пару дверей. Том постучался и, услышав ответное «входите, открыто», нажал на дверь. Он оказался в довольно похожем на его собственный кабинет. Еще когда он в первый раз пришел сюда, он подумал: «Почему они такие одинаковые?». Лишь картины на стенах отличались. Если у Тома были более сюрреалистичные мотивы, то у его коллеги преобладали классические произведения эпохи Возрождения.

– И снова здравствуйте, доктор Нуотс! — сказал Том, заходя.

— Приветствую. Как ваш разговор с братом? — дружелюбно принимал гостя толстяк. Том слышал, что тот несколько раз пытался сесть на диету и заняться спортом, но безуспешно.

— Ну, он заснул. Видимо рассказ мой был недостаточно интересен, — с некоторой иронией произнес Браун.

— Оу, впрочем, он сильно пострадал, так что это не удивительно. Вы по делу?

— Да. Есть какие-либо причины не переводить №53 обратно в его… — Том запнулся. — Комнату?

— Нет, можете спокойно вести его туда, — утвердительно сказал Нуотс.

— Хорошо, доктор. Тогда я немедленно этим и займусь, — гость уже собирался выходить, но тут его остановили.

— Том Браун… Том… Подождите, — чувствовалось, что человеком за столом пытается подобрать нужную интонацию, но ни одна из них его не устраивала. — Мы не можем оставить Майкла еще на некоторое время. Это не входило в наш бюджет и, — он вздохнул. — Мы немедленно отправим его обратно в хижину. Там уже навели порядок. Мы сотрем ему память обо всем произошедшем и внедрим ложные воспоминания о происхождении ран, — четко проговорил остаток предложения Нуотс.

— Ох, — Тому было несколько неприятно, что это происходит. Он хотел бы дорассказать историю, да и он так давно не видел Майкла. А теперь он даже вряд ли сможет с ним встретится. Ведь Майкл сменит имя, город, жизнь, а сам Том не имеет постоянного адреса и телефона. Поэтому все попытки со стороны Майкла тоже будут бессмысленны. Неприятное чувство росло в груди и постепенно начинало давить на горло. — Значит. Будет так. До свидания, доктор.

Том вышел из кабинета, оставив все свои переживания о случившимся там же. Сейчас у него есть работа, не оставляющая времени на эти раздумья.

______________________________________________________

— Хей, Исоб, я вернулся, — весело произнес Том. — У тебя же тут нет вещей, так? Тогда пойдем обратно в твою комнату. Только возьми эти тапочки, полы ведь холодные.

— Ура! — обрадовавшийся №53 принял обувь, и, надев ее, встал, уже готовый идти.

Они вышли в коридор и отправились в путь. Том за годы работы уже выучил все дороги и развики. Теперь полусферы стали для него лишь элементом декора. Они двигались вперед. Доктор Браун не совсем понимал, почему Исоба сразу не вернули в его комнату, а поместили в почти полностью пустую, да еще и в другом конце здания. Мерные тяжелые шаги и небольшое шарканье тапочек врывались в тишь каждого коридора, предупреждая о приближении людей растения и недвижимые двери. Внезапно шарканье прекратилось, а за ним через несколько секунд и тяжелые шаги.

Том обернулся и увидел, что №53 остановился, схватившись за голову. Мгновение спустя по коридору прокатился громкий стон.

— Исоб?!

— Д-д-друг… мне… мне плохо… я… что-то чувствую, я слышу крики. Мне страшно. Мне больно.

«Что… что происходит? — мысли хаотично носились в голове доктора. — Черт. Черт. ЧЕРТ!» — Том понял, где они находятся. Крик Исоба эхом разнесся по пустому пространству. Прямо за этой дверью, напротив которой стояло два человека, находилось своеобразное «кладбище». Здесь были уничтожены все предыдущие объекты. Медленно растоплены в кислотных ваннах. Конечно, они были уже мертвы. Но это было самой очевидной идеей для Тома. Страдания каждого объекта, вся их боль, были скоплены и запечатаны в этой стене, двери, помещении. И прямо сейчас на глазах у доктора Брауна стоял тот, кто открыл этот ящик Пандоры. Тот, кто заберет все содержимое себе, пусть и против своей воли.

Том бросился к кричащему и, схватив его, понес как можно дальше отсюда. Он бежал, оглушенный паникой, разрывающей его изнутри, но лицо его не выражало даже толики этого чувства. Наконец, истратив все силы, они остановились. Том положил №53 на холодный пол и тут страшное осознание пронзило его разум. Исоб был мертв. Он умер. Том сел на пол, не зная, что ему теперь делать. Его труд был уничтожен им же. Конечно, остались все записи, все образцы. Но он знал, что ему уже не предстоит работать с ними. Это была халатность. Огромная ошибка. Уже остывающее тело, которое было покинуто разумом, лежало перед ним. Мысли метались в его голове, словно ягоды в блендере. «Это конец. Меня отстранят, сотрут память, отправят куда-то, неизвестно куда. Но уж точно не на курорт. Скорее наоборот. Это конец». Том взялся за голову и сжал ее руками с двух сторон, словно желая раздавить ее своими руками, подобно прессу. Не преуспев в этом, он огляделся. Ослепленный паникой, он пришел прямо к своему кабинету. И тут между мыслью о том, что это конец, затесалась новая. Спасительная идея. «Это единственное, что можно сделать» — думал он, открывая дверь в свой кабинет. Здесь, в этой самой комнате, лежал ключ. Ключ, который он прямо сейчас использует для выхода из этой ситуации. Выдвинув первый ящик стола и вытащив оттуда уже бесполезные папки, его встретил он. Аккуратный «дамский» пистолет, с небольшой гравировкой в виде лозы на стволе.

– Так будет лучше, — прозвучало в пустой комнате перед следующим, теперь уже оглушающим, звуком.

______________________________________________________

Майкл проснулся на диване перед камином. Ему что-то приснилось, но уже спустя мгновение он никак не мог вспомнить, что же было в этом сне. На него также, как и вчера, смотрела величественная голова кабана, а за окном бушевала метель. Раны, полученные при перестрелке с участниками картеля, неприятно напоминали о себе. «Видимо, я несколько раз терял сознание уже в хижине, — думал он. — Еще и голова трещит». Повернувшись на бок, он вновь увидел книжные полки, наполненные самыми различными авторами и жанрами. «Что же, впереди долгая неделя», — он поднялся и подошел поближе к корешкам, скрывающим за собою захватывающие истории.

Оказалось, что неделя — это не так уж и долго. Погрязнув в готовке, уборке, колке дров, чтении, размышлениях перед огнем, Майкл даже немного огорчился, получив сообщение, что все готово. Странная тоска окутала его, когда он покидал хижину. Ему казалось, что за это время он прошел какое-то удивительное путешествие, а теперь же оно закончилось, так и не дав ответов на все его вопросы. Одинокая машина двигалась по шоссе, залитому ярким солнечным светом. На дорогу из окна выпало несколько непочатых пачек сигарет.

— Что бы Том сказал мне сейчас? — прозвучало в салоне, под аккомпанемент из медленной музыки, что затягивает слушателя в раздумья все глубже и глубже.