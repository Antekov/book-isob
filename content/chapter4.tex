\chapter{Рассказ}
\lettrine{М}{}айкл очнулся. Он проснулся. Яркий белый свет бил в глаза, ослепляя его. Прищурив глаза, он попробовал поднять руку, но почувствовал, что движение ее было затруднено, словно что-то было воткнуто в нее. Переведя взгляд, он увидел, что в предплечье вставлена игла. От нее шла тонкая трубка с непонятной жидкостью, что поступала от стоящей справа капельницы. Постепенно зрачки Майкла сужались, и он начинал различать окружающий его интерьер комнаты. Он лежал на больничной койке, слева стоял выключенный монитор на небольшой белой тумбочке. Все ее ящики были закрыты, а в верхнем была видна расцарапанная замочная скважина. Будто каждый раз ее открывали в страшной спешке и не успевали сосредоточится перед вставкой ключа. Вновь посмотрев направо за капельницей Майкл увидел ширму. Вся она была покрыта брызгами крови, будто неосторожный хирург случайно задел сонную артерию. Направив же свой взгляд вперед, Майкл увидел только свое отражение. Его одежду заменили на больничную пижаму, а волосы сбрили. «Голова трещит... ничего не понимаю. Я должен уходить. Неважно как», — думал он, рассматривая иглу в своей правой руке. Аккуратно взявшись за нее, он начал извлечение. Комнату оглушил звук случайно упавшей на пол иглы. «Проклятье!» – пронеслось в голове пациента. Медленно вставая с кровати, с напоминанием о произошедшем от ран, Майкл задумался о том, что идти просто так эквивалентно самоубийству. Недолго думая, он подошел к тумбочке и попытался открыть хотя бы один из ящиков, издавая как можно меньше шума. «Первый – закрыт. Второй – закрыт. Третий... Бинго», – Майкл достал из ящика скальпель. «Все еще эквивалентно, но все же лучше». Четвертый ящик также был закрыт. Отклонив обагренную кровью ткань ширмы, он шагнул вперед.

После столь светлого участка эта половина комнаты казалась ему утопающей в тьме. Но по мере продвижения вперед он все больше видел. Во второй половине помещения было довольно мало вещей: столы, компьютеры под ними, мониторы на них, стулья. На стенах висели плакаты с непонятными данными. Где-то в конце комнаты горела табличка «Выход», а рядом с ней стоял шкаф. Дверь и была целью Майкла. Он не думал о том, что он будет делать если выйдет. Единственная его цель – покинуть это место. Ведь ничего хорошего здесь произойти не могло, учитывая то, как его сюда «пригласили». Уже подойдя вплотную, комнату озарил яркий свет, а от двери послышался звук трясущихся ключей и поворота замка. «Да чтоб тебя!» – со смешанными эмоциями думал Майкл, уже залезая в тесный шкаф. В нембыло пыльно, все предметы, что были внутри, были покрыты пылью. Шкаф явно уже очень давно не использовался. Еле-еле разместившись среди банок и каких-то папок, Майкл затаил дыхание – замок щелкнул. 

– Как его состояние? – послышался голос сразу после открытия двери. 

– На удивление, все не так плохо. Да, он пострадал, но это не смертельно. Через две недели, при условии, что он будет принимать некоторые лекарства, он полностью избавится от всех проблем, – отвечал второй голос. 

– Это хорошо, очень хорошо. Я точно могу его увидеть? – с волнением спросил второй голос. Он казался Майклу очень знакомым, тем не менее сидящий в шкафу никак не мог понять, чей же он.

– Да-да, не волнуйтесь. Майкл Браун, выйдите из шкафа, пожалуйста, – настоятельно произнес второй голос. Прятавшийся был несколько удивлен, но, так и не придумав иного выхода, он спрятал скальпель за спиной, крепко сжав его в ладони.

– Хорошо. Я выхожу. 

– Да-да. Аккуратнее, пожалуйста, у вас могут разойтись швы. 

– Понял.

Майкл вышел из шкафа, держа руки за спиной. Перед ним стояло два человека. Оба они были в халатах, но у одного из них он был просто накинут на классический черный костюм. Майкл уже видел его. Буквально перед тем, как попасть в тот странный сон. Только теперь он наконец-то увидел его лицо. Майкл вспомнил голос. Он понял. Скальпель выпал из его руки, и он остановился в ступоре. Прямо перед ним стоял Том. Тот самый Том Браун. Его старший брат.

– Том? Что... что ты тут делаешь? 

– Тот же вопрос могу задать тебе, Майкл. Что ты забыл в той хижине и почему ты сейчас стоишь здесь, а не лежишь на койке? – укоризненно проговорил Том. 

– О, с памятью все хорошо. Была возможна временная амнезия... 

— Доктор Нуотс, вы не могли бы оставить нас на некоторое время? – спросил Том 

– А? Да, конечно. Дайте только я поставлю господину Брауну капельницу.

– Конечно, конечно...

Спустя пять минут дверь за Нуотсом закрылась. Том взял один из стульев, сел перед кроватью и скрестил руки на груди, откинувшись на спинку стула. 

– Так что? Что ты тут делаешь, братец? – спросил он. 

– Что я тут делаю? Что ты тут делаешь?! Ты исчез, я думал, что ты уже мертв, хоть и представить себе этого не мог! Ты...

– Спокойно, – перебил он Майкла. –  У меня были на это веские причины, уж поверь. Как ты здесь очутился, скажи пожалуйста. 

– Фуф... я приехал сюда, чтобы спокойно подождать, пока мне переделают документы, и я смог бы уехать в другой город и начать новую жизнь. Все ясно? – с недовольством в голосе произнес  младший брат.

– Ты все-таки смог уйти от преступности? Боже, Майкл, я так рад за тебя! Ты молодец! – радостно восклицал Том.

– Хм. Да, я и сам очень рад. Но ты не мог бы разъяснить, о радующийся за меня, что это, черт побери, было? 

– Ах. Ну да. Точно. Извини. Ты явно выбрал не лучшее место, чтобы спокойно переждать. Понимаешь, тут такое дело... – Том сбился.

– Какое? Я чуть не умер! Что такого может происходить в этом богом забытом месте, где даже связь ловит только при удачном стечении обстоятельств?! – Майкл явно был недоволен, но в глубине души он был очень рад вновь увидеть своего брата. Просто близкого человека. 

– Понимаешь, я не уверен, что тебе стоит это знать, – серьезным тоном проговорил Том. 

– Да ч... – Майкл резко поднялся на кровати и его бок пронзила адская боль. – Проклятье! – громко выругался он в воздух. 

– Я хочу знать. Что. Это. Было, – медленно сказал Майкл, смотря в глаза Тома.

– Хорошо. Но мне нужно отойти о посоветоваться с доктором о кое-чем. Если разрешат, я пошлю к тебе медсестру с кофе, – согласился Том, вставая и выходя из комнаты. Майкл даже не успел сказать что-нибудь ему вслед. Он уставился в потолок и погрузился в раздумья, граничащие с сонным бредом. Он видел, как он живет, как едет на работу на новом велосипеде, как неловко врезается в кого-то со стопкой бумаг... 

– Вот твой кофе, Майкл. Самому оказалось быстрее, хех, – сказал внезапно появившийся Том. 

– Благодарю. Без сахара? – спросил он, указывая пальцем на кружку.

– Конечно же. Пей, – призвал Том. – Я могу рассказать тебе все, но после ты подвергнешься экспериментальному методу стирания выделенного куска памяти. Нуотс сказал, что это практически точно безопасно. После этого ты проснешься в своей хижине с чувством прошедшего кошмара. Тебя устроит это? 

– И все будет, словно кошмар? Да. Конечно, да, – воодушевленно отвечал Майкл, аккуратно остужая кофе и грея руки о чашку. 

– Я расскажу тебе все и с самого начала. Поэтому история будет довольно длинной.

\begin{center}***\end{center}

В старой и потрепанной квартире, где воздух всегда оставался спертым, сколько ты ее не проветривай, лежал человек. Он лежал на кровати. Матрас ее прохудился и лежащий буквально утопал в нем. Скучающим взглядом он наблюдал, как по потолку ползет единственная муха. Движения ее были хаотичными, она бежала то вправо, то вперед, то взлетала и, покружившись под потолком, садилась обратно. Справа от кровати стоял небольшой кофейный столик. На нем валялась пустая бутылка из-под виски. За столиком была стена с одиноким окном, выходящим на серые и безрадостные здания. На строениях виднелись балконы, все сделанные на разный лад, что абсолютно убивало любого, кто обладал хоть толикой вкуса. Слева от окна были видны пятна от того самого виски, а на полу валялись осколки стекла от рюмки. Под опрокинутым стулом лежал дешевый деловой костюм. Ткань, из которой он был пошит, выглядела настолько неэстетичной, что внушала жалость. Где-то в глубине квартиры зазвонил телефон. Резкий звук оторвал человека от слежения за мухой. Он посмотрел на закрытую дверь, что отгораживала его от висящего в коридоре устройства. Вздохнув, он начал подниматься. Смотря под ноги, чтобы не наткнуться на осколки, он подошел к проему. Взявшись за ручку, человек понял, что телефон перестал звонить. Неуверенно стоя у двери, размышляя о том, чтобы вернуться в кровать, но уже начал разворачиваться. Но громкий звон вновь пронзил всю квартиру, будто бы проходя сквозь все поверхности, звуча прямо в голове.  Открыв дверь, он дошел до телефона и снял массивную трубку с док-станции, взяв висящую рядом ручку. Нервно прокручивая ручку в руке, он наконец услышал голос из трубки.

– Алло... алло? Это Том Браун? – с некоторым волнением вопрошал голос из трубки.

– Да, это я. Кто беспокоит? 

– Отлично-отлично. У меня есть к вам деловое предложение. Я понимаю, что у вас очень большая загрузка, но это должно вас заинтересовать. 

– Хмм... да, загружен я сейчас очень сильно, – Сказал Том, проходя на кухню, где в глаза ему вновь бросилась гора немытой посуды.

– Понимаете... это не совсем подходит для телефонного разговора. Когда мы могли бы встретиться? – с надеждой говорил некто звонящий. Том открыл холодильник. В нем стояли кастрюли, открыв одну из них, он почувствовал столь зловонный запах, что немного закашлялся. 
      
– Не люблю откладывать дела в долгий ящик. Сегодня в... – Том посмотрел на часы. «12:24», - горело на них. – Сегодня в час дня, в ресторане на улице Брейкдаун, – уверенным тоном сказал он.
      
– Конечно-конечно. Тогда до встречи.
Не успев даже попрощаться, Том услышал гудки. «Ну и ладно, – подумал он. – Надо бы немного привести себя в порядок», – поразмыслил он, потрогав щетинистое лицо. Он зашел в ванную комнату, что была, на удивление, очень чистой. Словно дверь, ведущая в нее, была порталом в уборную абсолютно другой квартиры. Том поднял глаза и увидел свое отражение в зеркале. Мешки под глазами, щетина, грязные волосы. В отличие от одежды, надетой на нем, он выглядел весьма помято. Том перевел взгляд на полочку при зеркале. Там стояли всевозможные кремы, дезодорант, пена для бритья и сама бритва. «У меня не так много времени. Минут 10 только идти до ресторана. Так что нужно поторопиться», – думал Том, уже нанося пену. Он работал очень быстро, но ни разу бритва не порезала кожу лица. Закончив, он вымыл голову и торопливо высушил волосы феном. Намазав лицо одним из кремов, он вновь уставился на свое отражение: «Так-то лучше», – подумал он. С зеркала на него смотрел уже довольно опрятный мужчина около двадцати пяти лет. Постояв еще немного, он смыл крем и вернулся в коридор. Зайдя в спальню, он было посмотрел на лежащий костюм, но быстро отмел эту идею. Даже в джинсах и футболке он будет выглядеть лучше, чем в этом творении. «Как я вообще смог купить такое?» – внутренне негодовал Том. Вновь медленно переступая все осколки, он прошел к шкафу и достал черные джинсы и белую футболку. Но секунду спустя Том отложил футболку, взяв черную водолазку. «Пожалуй, лучше так», – думал он.

Выходя и закрывая квартиру, он наконец задумался, кто же ему звонил. Но уже покинув подъезд, он отбросил эти мысли, ведь узнает это довольно скоро. 

\begin{center}***\end{center}

– Что-то у меня в горле пересохло, – сказал Том, покашляв.

– М? – Майкл протянул ему кружку, с почти закончившимся кофе.

– Да нет, спасибо. Я попрошу принести воды, – произнес Том, запуская руку за койку и доставая пульт. Нажав на кнопку с изображением стакана, он отпустил белую коробочку. Уже спустя минуты в комнату вошла девушка со стаканомводы. Майкл же молча лежал на постели, перебивать монолог собеседника он явно не собирался. 

– Спасибо, – Том начал пить. Закончив, он вернул сосуд девушке и, удовлетворено набрав воздуха, продолжил рассказ.

\begin{center}***\end{center}
\clearpage
{\begingroup
\ThisCenterWallPaper{1.0}{после 4 части.jpg}
\noindent
\endgroup}
\cleardoublepage
