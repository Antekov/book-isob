% \lettrine{}{} command typesets a drop cap in the document

% second brace set formats text placed inside as small caps
% (same thing as doing \lettrine{L}\textsc{orem ipsum ...}

\chapter{Ночь}
\lettrine{Н}{}очь уже окутала все вокруг своим шарфом из тьмы. Сегодня он не был украшен звездами или огромным брелком луны. Единственное, что можно было заметить и выделить — это снегопад, чья сила росла каждую секунду. В такую погоду приятно находиться дома и наблюдать из окна, как снег кружится в свете одиноко стоящего фонаря. Но не всем в эту ночь была доступна такая роскошь. На загородном шоссе в глубине леса одинокая машина вела нерав-ную борьбу с непогодой. Если заглянуть в саму машину, мы не увидим чего-то экстраординарного. Кожаные сидения, аккуратный вид, открытый бардачок, из которого чуть ли не выпадают игральные карты с салфетками. Над приборной панелью лежит открытая пачка сигарет, а в салоне играет медленная музыка, что затягивает слушателя в раздумья все глубже и глубже. Рядом с лицом водителя вспыхнуло пламя зажигалки и едкий табачный дым начал окутывать салон, который пропах этим дымом насквозь. Водитель продолжал движение, а снегопад уже перерос в метель.

Но водитель был не единственным в этом лесу. Прямо в это же время, сквозь метель, от которой немного спасали деревья, бежал кто-то, если это, конечно, можно назвать бегом. Каждый следующий шаг давался ему сложнее предыдущего, он спотыкался, но упорно продолжал движение. Он четко знал, куда он идет, но понятия не имел, что его будет там ждать – спасение или полная его противоположность. Путник продвигался все дальше и дальше, гонимый лишь ему одному понятным чувством, как вдруг он выпал на ровный участок земли. Сильный удар, что он должен был ощутить, был смягчен свежевыпавшим снегом. Но соприкосновение все равно вышло довольно болезненным. Он искал в себе остатки сил, чтобы подняться, но смог только повернуть голову. Он двинул её и увидел яркий свет, который становился все ярче и будто бы ближе. Страшная мысль появилась в его голове, но он отверг ее и вновь попытался встать. Свет прекратил свое движение и прямо сбоку от него появилась пламенно-красная точка. Постепенно в свете проявился силуэт человека и уже спустя несколько секунд он сел на корточки перед изнемогающим телом.

— Вот черт... Хей, ты еще здесь?

Ответа не последовало.

— Черт. Что делать? Он просто замерзнет здесь, если я его оставлю. Надо звонить в полицию. Или в скорую, — Он достал телефон, но сигнала не было.

— Проклятье! И что же? Мне придется брать его с собой? — он оглянулся, думая, что, может быть, можно свалить это дело на кого-то еще, но все, что он увидел — это снег, деревья и его собственная машина.

— Видимо, у меня нет другого выхода...

Он аккуратно взял его под подмышки и поволок к машине. Кое-как уложил его на заднее сиденье и сел за руль. Что-то ему подсказывало, что ночь будет длинной, а метель не закончится довольно долго.

Водитель завел машину и вновь двинулся в пучину ночи – дальше по шоссе. Он бросал взгляды в зеркало заднего вида, пытаясь узнать что-нибудь о ночном госте. Но ни одежда, ни лицо не давали ему ответов, а создавали больше вопросов. У пассажира не было куртки, что в этих краях считается чуть ли не самоубийством, ведь ночи очень холодны. Он был в классической белой рубашке, что была ему немного велика, в синих джинсах и кедах. Вся его одежда и волосы были покрыты снегом, что медленно таял прямо на кожаные сидения. Местами, снег покрывал его словно тонкой коркой. «Его наряд подошел бы какому-нибудь теплому городу в середине лета, но в этих краях он не мог выйти на улицу ночью в таком!» — думал водитель. Разум его разрывался от вопросов, а предчувствие странно молчало. Он надеялся, что узнает все, когда доедет до пункта назначения.

\begin{center}***\end{center}

Шоссе закончилось, теперь осталось только что-то наподобие дороги. В скором времени машина остановилась, и водитель вышел из нее. Засунув руки в карманы и попытавшись сжаться, лишь бы укрыться от вездесущего снега, он двинулся в сторону едва видной бревенчатой хижины. Он подошел к двери и достал ключи из кармана — руки уже тряслись от холода и ключ никак не хотел попасть в замочную скважину. Дверь, наконец, открылась и он зашел внутрь.  Воздух внутри хижины был спертым, отчего казалось, что мрак внутри нее был густым и продвигаться в нем было непросто. Войдя внутрь, водитель первым делом достал зажигалку, чтобы хоть как-то освещать себе окружение. Он увидел выключатель, но, нажав на него, ничего не произошло.

— Ах, ну да. Точно. Генератор, — проговорил он себе под нос и вновь вышел на улицу. Обойдя здание вокруг, он нашел дверь, а за ней обнаружил генератор, выглядящий довольно старым, но надежным. На удивление, он заработал практически сразу и, запирая дверь, он уже видел, как немного света проникает через окно. Вернувшись ко входу, он обернулся к машине, раздумывая о том, что же ему делать с пассажиром. Спустя мгновения он стоял около машины и открывал дверь, чтобы затащить своего вынужденного гостя в дом. Водитель уложил его на диван, стоящий посередине комнаты, и оглядел её при свете.

Она не могла похвастаться обильным количеством вещей: диван, пара стульев, небольшой столик перед диваном, книжные полки, камин и чучело головы кабана прямо над камином. В противоположной от входа стены было две двери. С потолка свисала старая лампочка, что еле-еле давала необходимый минимум света. Водитель вновь взглянул на своего гостя при слабом свете он мог уже лучше рассмотреть его. Он наблюдал за ним минуты две и никак не мог понять, что же так сильно вводит его в ступор. Что не так с этим человеком, лежащим на диване? Почему, когда он смотрит на него, он начинает чувствовать диссонанс и еще одно странное чувство, которое он не может объяснить себе самому? Он простоял так еще пару минут, после чего он, все-таки, вышел из своеобразного транса и вновь огляделся. Обратив внимание на двери, он подошел к левой двери и зашел туда. 

Это была кухня с громадным гарнитуром, что зани-мал большую её часть и казалось, что в его ящиках и шкафах можно сложить столько, что обычному человеку столько не набрать. Также в кухне были два стула и небольшой стол, на котором стояла свеча, она явно использовалась множество раз и теперь от нее осталась лишь маленькая горка воска с торчащим фитилем. Впрочем, это не сильно волновало водителя. Он подошел к шкафу и, открыв его, увидел полностью заставленные тушенкой и макаронами полки. В остальных шкафах также лежали долгохранящиеся продукты: гречка, рис, консервы и вода. Водитель вышел из кухни довольным. Он до сих пор был в куртке, что было не очень удобно. Он снял ее и положил ее на своего гостя. Он уже собирался идти осмотреть вторую комнату, но возле правой двери в его взгляд бросились дрова. Тогда он решил, что лучше сперва обеспечить дом источником тепла, ведь в хижине было, мягко говоря, прохладно. Он начал разжигать огонь в камине и уже спустя некоторое время тепло начало постепенно окутывать комнату. Довольный собой, он вновь пошел к правой двери, но только взявшись за ручку он услышал сдавленный стон позади себя. От неожиданности он вздрогнул и развернувшись увидел, как его куртка медленно сползала с уже открывшего глаза гостя. Водитель не был параноиком, но что-то заставило его отвести руку за спину и положить ее на рукоять пистолета, что был аккуратно спрятан между его телом и джинсами. 

Стон вновь прозвучал в тиши комнаты. 

— Приятель, ты как? - спросил водитель

— Я... я. Я в порядке. Я можно попросить вода?

— Попросить воды? Да, конечно, сейчас, — сказал водитель, направляясь в сторону кухни. Краем глаза он заметил, как странный гость подбирает куртку и накрывается ей, словно одеялом. Наливая воду в стакан, водитель никак не мог отделаться от странного чувства и мыслей о речи того, кто лежит прямо за дверью. Его искушало желание просто избавиться от него, но при этом что-то останавлива-ло, отвергало эту идею в его сознании. 
Вернувшись в зал, он увидел, что его гость уже сел, но продолжал держать куртку словно щит, который должен защитить его от всего. Весь снег на нем растаял, и только влага осталась на одежде и волосах. Гость внимательно смотрел на огонь, словно пытаясь разглядеть в нем что-то.

— Вот, возьми, – сказал водитель и протянул стакан воды.

— Спа...сибо... друг, — промолвил сидящий и взял стакан воды.

«Друг? Он сказал друг? Да кто он? Что с ним? Больной или, может, сумасшедший? Но в этих краях нет никаких психиатрических больниц. Здесь кругом лес, километров на сто», – думал водитель пока его гость пил воду, дрожащими руками держа стакан. И тут водителя словно молнией ударило понимание того, что же его так настораживало: «Его руки. Они были абсолютно здоровы на вид, как и другие открытые взору участки кожи. Но если вспомнить то, как много снега было на нем, нельзя сказать, что он пробыл на улице совсем немного. Тогда почему на его коже нет никаких следов обморожения? Она лишь немного бледна! Она выглядит абсолютно нормально! Так, спокойно, от паники до безумия лишь шаг. Должно быть логическое объяснение. Спокойно» — так думал водитель. Напряжение в комнате росло, водитель ощущал, что его левый глаз вновь начал дергаться, а рука вновь вернулась за спину, сжимая рукоять пистолета все сильнее. Тем временем гость допил и протянул стакан, попутно благодаря и спрашивая водителя:

— Как тебя зовут?

— Меня? — переспросил водитель. — Меня зовут Майкл. Как зовут тебя? — он не мог понять, что несут за собой действия его гостя. «Он ведет себя, как ребенок. Он словно не помнит, что произошло или хочет, чтобы я так думал. Мне надо попытаться узнать, как можно больше. Мне бы не хотелось убивать его: слишком много проблем с избавлением будет, да и лучше «не шуметь тут».

Его раз-мышления были прерваны неожиданно громким ответом:

— Я Исоб! Рад с тобой познакомиться! — он говорил эти фразы с таким восторгом, словно ожидая награды за каждое из сказанных слов. Это же ожидание Майкл увидел в его глазах.

В секунду напряжение ушло и комнату заполнило что-то теплое. Майкл не мог объяснить, что за скачок он наблюдал, но тело его расслабилось и все боли покинули его. Он убрал руку с пистолета и присел на стул, стоящий сбоку от стола. В камине изредка трещали дрова, нарушая полную тишину, что окутала помещение. Никто из присутствующих не хотел говорить. Исоб внимательно смотрел в огонь, а «друг» наблюдал за ним, погруженный в раздумья. Разум Майкла гудел, мысли нехотя сменялись, а вопросы словно терялись в пучине сознания. Уже привычным движением Майкл достал пачку сигарет из нагрудного кармана рубашки и зажигалку из джинсов.

Он взял длинный белый цилиндр с оранжевым осно-ванием в тонкие изящные руки. Такие руки ценились бы у актера или шоумена – они так и притягивали бы взгляд зри-теля. Но для Майкла они не были чем-то особенным, он привык получать комплименты в их сторону, но слушал их с тихой усмешкой себе под нос. Никто не знает, что было свершено этими руками. Сколько они прошли и, наверное, пройдут, если сейчас все обернется крахом. Он не может этого допустить. Рисковать сейчас - безумие. Упускать этот шанс стать обычным человеком – бред. Майкл понимал это. тем не менее, на диване, прямо перед ним, сидел человек, что может привести его к тому самому концу. Человечность не всегда является другом. С такими мыслями сидел Майкл, поднося сигарету с горящей зажигалкой ко рту. Дым возносился к потолку, медленно исчезая, растворяясь в нем. «Сколько пачек я выкурю до того, как смогу покинуть это место?» — думал Майкл, делая очередную затяжку. Он пе-ревел взгляд на окно — метель все еще бушевала и, каза-лось, конца этому не будет.

— Зачем? — разрушив ауру тишины, спросил Исоб. — Зачем ты это делаешь?

— Что именно? Курю? 

— Да! Зачем ты это делаешь? Ты... ты умираешь.

— Может и так, — отвечал закуривающий водитель. — Но пока это помогает мне, я не собираюсь бросать. Все мы умираем, но с разной скоростью. Я просто добавляю в свою жизнь нечто, что снимает стресс и укорачивает отве-денное мне время, а люди, что не курят, сокращают свою жизнь стрессом, снять который времени нет, а то, что хоть как-то помогает расслабится – для них слишком «вредно», – с некоторой издевкой сказал Майкл, стряхивая пепел с сигареты на листок бумаги, лежащий на столе.

– Это неправильно. Ты не должен курить, — строго сказал Исоб. Его лицо выражало истинную решимость, будто он готов был вырвать сигарету изо рта собеседника и бросить ее прямо в огонь. Он уже никак не походил на беззащитного взрослого ребенка, что сидел на этом диване, накрывшись курткой. 

— Правильно или неправильно. С чего ты решил, что именно \textit{ты} можешь судить об этом? Ты понятия не имеешь, кто я. Ты не вправе судить меня. Я, вообще-то, спас твою жизнь! — раздраженно, чуть ли не агрессивно отвечал Майкл. 
Обычно он не был таким, но последние события и действия весьма тяжело дались ему. Что бы он не говорил, а стресс преследует его, хоть он и пытается всячески подав-лять все проявления оного, но он все равно находит пути наружу – через злость и раздражение. Внезапно Майкл по-нял, что боль вновь вернулась и окатила его своей волной. А напряжение вновь обуяло в комнату. 

— А я спасаю твою. \textit{Отдай мне сигарету}, — медленно промолвил Исоб, выговаривая каждое слово по отдельности. В его голосе звучал холод, словно он втянул в себя весь снег с округи, преобразовав в речь. Майкл отнес сигарету от рта, зажав ее между средним и указательным пальцем. Он остолбенел, почувствовав исходящую угрозу от своего гостя. Он видел влиятельных людей, он работал с ними. Некоторые пугали его, некоторые внушали уважение. Но \textit{он} — он был другим. Майкл не мог понять, что же за чувство окутало его. Это не был страх — это было нечто \textit{другое}. Словно все доступные ему чувства взяли и замешали в одном огромном котле и сейчас ему дали испить стакан этого отвара. Исоб требовательно протянул руку и Майкл послушно положил в нее сигарету. Мгновение спустя сигарета уже горела в пламени, под сопровождение, состоящее из треска поленьев.
\clearpage
{\begingroup
\ThisCenterWallPaper{1.0}{после 1 части.jpg}
\noindent
\endgroup}
\cleardoublepage
